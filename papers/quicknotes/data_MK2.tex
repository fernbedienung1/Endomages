\documentclass[10pt, a4paper]{article}

\title{dataAcquirement V2} 	
\author{Max}

\pdfinfo{
	/Title		(dataAcquirement V2)
	/Subject	(V2 - smoke generation)
	/Author		(M)
	/Keywords	(Endoscope, laparoskope, 42, surgical smoke)
}

\begin{document}
	\maketitle 

	\section{Preset}
		\subsection{Organs}\label{subsec:Organs}
			\paragraph{Organ order \\}
				slaughter-date:	21.12.2018 8:00 - 12:00 \\
				pickup date:	21.12.2018 - 13:00 \\
			\paragraph{location\\}
				Michael Kleiber GmbH \\
				Saarlandstr. 23 \\
				87700 Memmingen \\
			\paragraph{organs}
				\begin{itemize}
					\item Liver
					\item Gallbladder 
					\item Stomach
					\item Pancreas 
					\item Intestines 
					\item 1L Blood
				\end{itemize}
				All the organs are withdrawn form a domestic Pig.

		\subsection{Tools}\label{subsec:Tools}
			\paragraph{Cameras}
				\begin{itemize}
					\item \textbf{Wolf} - 1CCD Endocam 5520\\
						Bronchoscope  
					\item \textbf{Wolf} - 3D-HD-Endocam F1014\\
						Stereo Laparoscope
					\item \textbf{IDS} vi-3240cp-c-gl \\
						Industrial Camera with endoscope Optic
				\end{itemize}
			\paragraph{Frame Grabbers}
				\begin{itemize}
					\item \textbf{Epiphan Video} - DVI2USB3.0 \\
						DVI Frame grabber
					\item \textbf{reflecta} - USB Video Grabber \\
						Composite Frame grabber	
					\item \textbf{Terratec} - G1 \\
						Analog Frame grabber
				\end{itemize}

			\paragraph{Cauter}
				\begin{itemize}
					\item \textbf{Aesculap} - Caiman 5\\
						Electro cauter
					\item \textbf{Aesculap} - GN200\\
						High Frequency Generator	
				\end{itemize}

		\subsection{Environment}\label{subsec:Environment}
			Endoscopic Test box - build to be close to a Laparoscopic training simulator. \\
			
		\section{Method}\label{subsec:Method}
		\subsection{scene}
			Videos will be captured with 3 different Laparoscope\ref{subsec:Tools} \\
			The Cameras as well as the Cauter will be inserted into the Test box through a neoprene layer, which simulates the insertion through the Patients Skin. \\
			The Organs are going to be placed inside of the Test box all at once to create a setup similar to a real visceral surgery.\\
			During the Video Capturing process multiple organs are about to be incinerated. \\	
			Preliminary tests for a good camera setup will be performed on a extra liver, which is ought to be organized previously. \\
		
		\subsection{setup types}
			Multiple setups of the scene will be taken:
			\begin{itemize}
				\item Static Setup and Dynamic \\
					(moved and fix Camera Position)
				\item 3 different Cameras (see \ref{subsec:Tools})
				\item with and without active Cauter 
				\item with and without Tools in the operational Area
			\end{itemize}
			All of the created Video Clips should have a length of 30 seconds.\\

		\paragraph{Dynamic Setup}
			The movements within the dynamic part imitate real camera movements in surgeries, which mostly include movements like:\\ 
			\begin{itemize}
				\item \textbf{90 degrees right/left around the Patients Axis} 
				\item \textbf{30 degrees right/left around the laparoscopies Axis}
				\item \textbf{Zooming in and out a distance of ~5-7 cm}
			\end{itemize}
		\paragraph{output files \\}
			the Pattern:	\textbf{CAMERA\_GRABBER\_TIME\_SETUP.type \\}
			Will be used for the naming of the files for identification.

	\section{Timing}
		\paragraph{9:15} departure at NewTec with Company Car
		\paragraph{10:00} loading Car at HSU with Equipment and return to NewTec.
		\paragraph{11:00} Setup of environment at NewTec ( Internet / etc.)
		\paragraph{11:30} departure to Memmingen for Organs.
		\paragraph{\\ \textit{further planning depends on Pickup time in Memmingen\\}}
		\paragraph{14:00} latest expected Arrival time at NewTec and start of data-acquirement.
		\paragraph{17:30} finalization of Tests and packing of Equipment.
		\paragraph{18:00} departure to HSU
		\paragraph{19:00} Company Car return to Carpool, Cleanup of NewTec Laboratory
		
\end{document}
