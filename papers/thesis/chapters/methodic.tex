\fuck{Check this section again with ANE-comments}
%%%%%%%%%%%%%%%%%%%%%%%%%%%%%%%%%%%%%%%%%%%%%%%%%%%%%%%%%%%%%%%%%%%%%%%%%%%%%%%
%% FUNDAMENTALS
%%%%%%%%%%%%%%%%%%%%%%%%%%%%%%%%%%%%%%%%%%%%%%%%%%%%%%%%%%%%%%%%%%%%%%%%%%%%%%%
\section{Fundamentals}
	Surgical smoke is different to ''normal'' smoke because of its contents and their composition as given by the paper \cite{contentOfSmoke}. Therefore the generation of the test data needs to be performed with a special focus on the realism of the smoke and haze. Also not only the cauterized organs and their humidity are important to the aspect of the smoke. According to experts also the method as well as the duration of cauterization is an important influence on the smokes quality.  \newline

	Training situation for doctors-to-be include exercises in laparoscopic simulations, so called phantoms. These phantoms are a emulation of a real laparoscopic surgery and 
	The paper \cite{MIS_host} introduces a learning method for expectant surgeons to practice without living patients, be it human or animal. Since these surgeries are expensive, time consuming and moreover they need to be authorized by governmental or veterinarian institutions in the European Union. \newline

	The data generation of surgical smoke and artefacts requires the use of organic material that can be cauterized. Porcine visceral organs are very similar to human visceral organs which the paper \cite{swineModel} postulates and are therefore a appropriate candidate for the test setups. \newline

	Laparoscopic trainers are often used in eduction and training of future surgeons. Therefor a source for the setup of these kind of trainers and how they should look like is taken from the papers \cite{MIS_host} and \cite{MIS_trainer}. \newline

%%%%%%%%%%%%%%%%%%%%%%%%%%%%%%%%%%%%%%%%%%%%%%%%%%%%%%%%%%%%%%%%%%%%%%%%%%%%%%%
%% PREQUISITES
%%%%%%%%%%%%%%%%%%%%%%%%%%%%%%%%%%%%%%%%%%%%%%%%%%%%%%%%%%%%%%%%%%%%%%%%%%%%%%%
\section{Test Perquisites}
		In order to capture videos that are related to real surgeries multiple factors need to be considered. Since not all animals have tissue similar to humans a suitable host needs to be selected.\\

	\paragraph{\textbf{Mice} \textit{(Mus musculus)}\\} have a quite high similarity to the human body, but problems occur in the availability and the handling of these organs. Therefore Mice's organs drop out, since they are expected to have a bad handling due to their small size. \newline
	\newline
	\paragraph{\textbf{Porcine Organs}\textit{(Sus scrofa domesticus)}\\} fit better for this purpose, but especially their internals are not very commonly consumed. For closer testings it is expected that porcine organs are necessary, while for first setup tests or technical evaluations other organs might be sufficient.\newline
	\newline
	\paragraph{\textbf{Bovine Organs} \textit{(Bos primigenius taurus)}\\} are the most commonly available visceral organs, since they are often used in culinary way. Tripes or livers especially. Although their similarity to human internals is not that high, their advantages in availability and usability makes them a good choice for first experiments.\newline
	\newline	 \\
	As given in \cite{swineModel}, the common domestic pig has already often been used as the major species for animal testing of both pharmaceutical and surgical experiments. Due to the fact that pigs have a lot of similarities to the human body, e.g. the resemblance of their cardiovascular systems or the digestive systems.\newline

	Concluding to this, the data should contain a set of video footage that is based on the cauterization of porcine internals.

	\paragraph{Organs\\}
    	The most laparoscopic surgeries are performed in the area of the abdomen, so the most reasonable things to use as target objects for the cauterization would include:
		\begin{itemize}
			\item Gallbladders ( Vesica biliaris),
			\item Livers ( Hepar ),
			\item Stomaches ( Gaster ),
			\item Gastrointestinal tracts ( Tractus digestorius ).
		\end{itemize}

		The selection of organs depends on similar choices as the selection of animals. The decision is based on their availability and usability, countering to real surgeries like liver resections or cholecystectomies.\newline
		Another point is the internal moisture that is given by mucous membranes, their cauterisation leads to fog and steam. This issue needs to be considered since moisture is expected to create foggier images that therefore are closer to real surgeries within wet environments.\\
\newline
		This moisture occurs in forms like Blood, lipid and other body internal fluids. By heating up these liquids different forms of steam, fog and stains are created which all have different aspects. Based on this information a insertion of Blood, etc. into the simulation needs to be considered. \\ 
	\newline
       Body internal humidity has impact on the distribution of smoke and staining within the surgical area (e.g. a \gls{Pneumoperitoneum}). 

	\paragraph{Hardware \\}
        The hardware setup includes:\\
        \cmt{Table to big - fix required}
        \begin{table}[H]

            \centering
            \label{tab:Cameras}
            \begin{tabular}{l|l|l|l}	%all left aligned
                \textbf{Company} & \textbf{Model} & \textbf{Type} & \textbf{REF}\\ 
                \hline
                Wolf & Endocam & Bronchoscope & \cite{endocam} \\		%1CCD Endocam 5520 
                Wolf & 3D-Endocam & Stereoendscope & \cite{} \\		%3D-HD-Endocam F1014 
                IDS & ui-3240cp & 1280x1024/60FPS& \cite{} \\ 
                Nikon & D3400 & 6000x4000 & \cite{dslr} \\
                Logitec & C270 & 1280x720 & \cite{webcam} \\
                Alcor Micro & SJ00446 & 1280x170 & \fuck{NONE} \\
                \hline
                Epiphan & DVI2USB3.0 & 1920x1080/30FPS & \cite{epiphan} \\
                \hline
                Aesculap & Caiman 5 & Vessel sealer & \cite{sealer} \\
                Aesculap & Letrafuse & RF-Generator & \cite{hfgen} \\
            \end{tabular}
            \caption{Capturing Hardware}
        \end{table}

    The Cameras are inserted into a self-made laparoscopic trainer\ref{fig:phantomI}.
        
    \begin{figure}[H]
        \centering
        \includegraphics[scale=0.06]{pics/setups/Laparoscope_box.jpg}
        \caption{self build Laparoscope phantom}
        \label{fig:phantomI}
    \end{figure}
    
    With penetrable neoprene cloth to simulate skin. \ref{fig:phantomII}.

    \begin{figure}[H]
        \centering
        \includegraphics[scale=0.06]{pics/setups/Laparoscope_box_cover.jpg}
        \caption{neoprene cloth with holes form tool insertion}
        \label{fig:phantomII}
    \end{figure}
	
	\subsection{Required Data}
		
	The resulting data needs to fulfill several requirements, mostly concerning the occurrence and mixture of noise the as well as the quality of the images.\newline
	Investigated noises are:
	\begin{itemize}
		\item Surgical Smoke,
		\item Haze,
		\item Staining of
		\begin{itemize}
			\item Blood,
			\item Water,
			\item Lipids.
		\end{itemize}
	\end{itemize}
	
    The generation of data is based on video data given by the \textit{Kreiskrankenhaus Günzburg}.
	These videos have been used as a template to generate own Data corresponding to the given real videos.
    The videos \fuck{video REF here} \cmt{screenshot form vid}, show prevalent movements of surgeries which are reproduced as close as the setup and experience allow.\\
    \\
    As the generated Data will used as base for the code generation additional requirements besides the realism aspect need to be met.\\ 
    One of these conditions is the slow rise of noise, which is expected to support the algorithm creation. By e.g. showing the thresholds for recognition or reconstruction.
	

    \newpage
%%%%%%%%%%%%%%%%%%%%%%%%%%%%%%%%%%%%%%%%%%%%%%%%%%%%%%%%%%%%%%%%%%%%%%%%%%%%%%%
%% LABORAUFBAUTEN
%%%%%%%%%%%%%%%%%%%%%%%%%%%%%%%%%%%%%%%%%%%%%%%%%%%%%%%%%%%%%%%%%%%%%%%%%%%%%%%
	\subsection{Test Setup}
        Within the project multiple setups have been created. With each iteration of prototypes the quality of realism and the coverage of the requirements improved.\\
        \newline
        Since laparoscopes, cauters and other medical equipment tend to be rare and expensive within causal working environments the first prototype \ref{sec:Prototyp} focuses on a way to avoid this kind of professional equipment. With minor success in this belonging, a more realistic setup is created this Acquirement setup \ref{sec:Acquirement} uses real medical devices with have been provided by Prof. Dr. Alfred Franz and the \textit{DKFZ \gls{DKFZ}}. \\

%%%%%%%%%%%%%%%%%%%%%%%%%% FAKE DATA %%%%%%%%%%%%%%%%%%%%%%%%%%%%%%%%%%%%%%%%%%%%%%%%%%%%%
        \subsubsection{Prototype} \label{sec:Prototype}
            The first prototype was created to identify to most likely issues that could interfere with the data generation. \\
            Prototype \texttt{\textbf{MK\_I}} is assembled from common items, out of pure convenience. Since the organization of real, fresh porcine visceral organs as well as the acquisition of corresponding medical equipment created various organizational and bureaucratic hurdles. \\
            \newline
            The table \ref{table:usedHW_I} in combination with pictures \ref{fig:MK_I_schematic} and \ref{fig:MK_I_real} show the used Items and their constellation.
            \newline 
            
            \begin{table}[H]
                \centering
                \begin{tabular}{c|l|l}	
                    \textbf{REF} & \textbf{Device} & \textbf{Description} \\ 
                    \hline
                    \textbf {1} & Raspberry PI & controlling element\\
                    \textbf {2} & DSLR & Main Camera\\
                    \textbf {3} & Webcam & USB-Camera\\
                    \textbf {4} & Endoscope & close up shots\\
                    \textbf {5} & Environmental sensor & Temperature/humidity\\
                    \textbf {6} & Soldering Iron & improvised cauter\\
                    \textbf {7} & LED light & controlled light source\\
                \end{tabular}
                \label{table:usedHW_I}
                \caption{Used hardware in first prototype \\ numbers correspond to \ref{fig:MK_I_schematic} and \ref{fig:MK_I_real}}
            \end{table}

            The phantom is shown in the picture \ref{fig:MK_I_real}. These pictures where taken during the process of video generation. 
            The schematic \ref{fig:MK_I_schmatic} shows the composition of the single components, with their connections to the internals. \\
            An example of the created data is shown in picture \ref{fig:MK_I_result} \\
            \newpage

%%%%%%%%%%%%%%%%%%%%%%%%%% Pictures of Test Setup %%%%%%%%%%%%%%%%%%%%%%%%%%%%%%%%%%
		    \begin{figure*}[hp]
                \begin{subfigure}[b]{\textwidth}
                    \centering
                    \includegraphics[scale=0.34]{pics/setups/setup1_schematic.png}
                    \caption{\textbf{Schematic} \\
                        Plan for fist Prototype, with the intention to evaluate Cameras and Tools. \\ 
                        (1)Raspberry PI B3+, (2)DSLR, (3)Webcam, (4)Endocam, (5)Temperature/Humidity Sensor, (7)LED-light source. \\
                        \textit{Devices correspond to table \ref{tab:usedHW_I}.}
                    }
                    \label{fig:MK_I_schematic}
                \end{subfigure}

                \hspace{1cm}

                \begin{subfigure}[b]{\textwidth}
                    \centering
                    \includegraphics[scale=0.4]{pics/setups/setup1_real.png}
                    \caption{\textbf{Real} \\ 
                        Real setup based on Schematic\ref{fig:MK_I_schematic}. \\
                        (1)Raspberry PI B3+, (2)DSLR, (3)Webcam, (4)Endocam, (5)Temperature/Humidity Sensor, (7)LED-light source. \\
                        \textit{Devices correspond to table \ref{tab:usedHW_I}.}
                    }
                    \label{fig:MK_I_real}
                \end{subfigure}

                \hspace{1cm}

                \begin{subfigure}[b]{\textwidth}
                    \centering
                    \includegraphics[scale=0.2]{pics/setups/setup1_result.png}
                    \caption{\textbf{Result} \\ 
                        Images taken form the videos produced with Setup \ref{fig:MK_I_real}. \\
                        (1)DSLR-output, (2)Webcam-output, (3)Endocam-output.
                    }
                    \label{fig:MK_I_result}
                \end{subfigure}

                \caption{fist Surgery phantom Prototype}
                \label{fig:MK_I}
            \end{figure*}

            \newpage

            \paragraph{lessons learned}
                Working with the first Prototype, resulted mostly in technical experience in the field of bench tests. \\
                The evaluation of cameras and cauters had the conclusion that a generation of realistic data will only be possible with professional tools. \\
                While the test of the laparoscopes phantom was quite successful, it created the need for further improvements, like a way to insert tools through a skin-like material. In all other means the phantom was suitable designed and fulfilled its purpose quite convenient. \\
                \newline
                The Prototype-setup therefore concluded that the given cameras are not sufficient to meet the requirements of the required data because:
                \begin{itemize}
                        \item their usage does not allow to perform the requested movements. 
                        \item the produced smoke was created using a soldering Iron. \\
                            Its Temperature curve is different to medical Thermokauters, which heat up way faster. \\
                            Besides this most cauterization is performed using electrocauters. This type of cauters will create different forms and composition of  smoke than a thermal cauter.
                \end{itemize}

            First tests using the shown setup led to the assumption that the used cameras are not capable of creating realistic videos. \\
            The smokes physical aspect is depended to the form of its creation. Smoke created by lasers will look different than Smoke created by a incineration using electro-cauters or thermal-cauters.

            \newpage    % Check if that makes it nice

%%%%%%%%%%%%%%%%%%%%%%%%%% REAL DATA %%%%%%%%%%%%%%%%%%%%%%%%%%%%%%%%%%%%%%%%%%%%%%%%%%%%%
        \subsubsection{Acquirement Setup} \label{sec:Aquirement}

        
            The table \ref{table:usedHW_II} in combination with pictures \ref{fig:MK_I_schematic} and \ref{fig:MK_I_real} show the used Items and their constellation.
            \newline 
            
            \begin{table}[H]
                \centering
                \begin{tabular}{c|l|l}	
                    \textbf{REF} & \textbf{Device} & \textbf{Description} \\ 
                    \hline
                    \textbf {1} & Laparoscope & capturing element\\
                    \textbf {2} & Neoprene & penetrable skin \\
                    \textbf {3} & Optic & inserted into trainer\\
                    \textbf {4} & Cauter & active tool\\
                \end{tabular}
                \label{table:usedHW_II}
                \caption{Used hardware in first prototype \\ numbers correspond to \ref{fig:MK_II_schematic} and \ref{fig:MK_II_real}}
            \end{table}

			\cmt{professional Cauter} the cauter detemines the consistens of the produced smoke. Therefore it is one of the mission critical devices in the acquirement setup.\\
			\cmt{professional Endoscopes} are un inevitable since only Laparoscopes capture laparoscopic pictures. \\ 

            \newpage


%%%%%%%%%%%%%%%%%%%%%%%%%% Pictures of Acquirement Setup %%%%%%%%%%%%%%%%%%%%%%%%%%%%%%%%%%
		    \begin{figure*}[hp]
                \begin{subfigure}[b]{\textwidth}
                    \centering
                    \includegraphics[scale=0.4]{pics/setups/setup2_schematic.png}
                    \caption{\textbf{Schematic} \\
                        Schematic of the data acquirement setup \\
                        (1)Endoscope, (2)Neoprene, (3)Optic, (4)Electrocauter
                    }
                    \label{fig:MK_II_schematic}
                \end{subfigure}
                \hspace{1cm}
                \begin{subfigure}[b]{\textwidth}
                    \centering
                    \includegraphics[scale=0.4]{pics/setups/setup2_real.png}
                    \caption{\textbf{Real} \\ 
                        acquisition progress with Stereo-Laparoscope \\
                        (1)Endoscope, (2)Neoprene, (3)Optic, (4)Electrocauter
                    }
                    \label{fig:MK_II_real}
                \end{subfigure}
                \caption{surgery phantom Setup}
                \label{fig:MK_II}
            \end{figure*}

            \newpage
                
            \newpage

	\subsection{Produced Data}
        The produced data can be put into 3 different categories. \\
        \paragraph{Static shots \\}
            The static shots are performed within a not moving environment.\\
            They show a scene in which the mentioned noises of surgeries slowly rise. This Video data can be used to test the Algorithm until a breaking point.
            E.g. steam rises slow within the environment and then starts to stick to the camera. The breaking point is then reached when the algorithm has no chance left to reconstruct the original image through the steam. \\

        \paragraph{Dynamic shots \\}
            The dynamic setups show scenes that include the most common movements in surgeries, \\ with are:
            \begin{itemize}
                \item rotations around the tool axis (-45 to 45 Degrees),
                \item insertions (push / pull) movements,
                \item rotations around the patients axis (-60 to 60 Degrees).
            \end{itemize}

        \paragraph{Realistic shots \\}
            Realistic shots do not follow a special form of movement pattern, but focus on a clear sight on the cauter and the parts that are incinerated. \\
            They combine all \textit{Dynamic Movements} mentioned above. But are in contrast to the Dynamic shots these movements are not intended. They result from the goal to keep the Endoscope focus on the operation area, the tools and the Organs that the surgeon works with. \\
            \newline
            This setup results in data that is very close to real surgery-footage, compared to the real data \cmt{Ref Günzburg} given by Dr.Widmaier.\\

        \begin{figure*}[hp]
%%%%%%%%%%%%%%%%%%%%%%%%%%%%%%%%% FIRST ROW
            \begin{subfigure}[b]{0.3\textwidth}     % eindritteln der Textfläche für plazierung neben einander
                \centering
                \includegraphics[width=\textwidth]{pics/ownData/dynamicSmoke_s0.jpg}
                \caption{\textbf{dynamic Smoke}  \\
                    Dynamic smoke 1
                }
                \label{fig:dynamic_1}
            \end{subfigure}
            ~ 
            \begin{subfigure}[b]{0.3\textwidth}
                \centering
                \includegraphics[width=\textwidth]{pics/ownData/rotation_s0.jpg}
                \caption{\textbf{more dynamicsmoke} \\ 
                }
                \label{fig:dynamic_2}
            \end{subfigure}
            ~
            \begin{subfigure}[b]{0.3\textwidth}
                \centering
                \includegraphics[width=\textwidth]{pics/ownData/stain_s0.jpg}
                \caption{\textbf{even more dynamic smoke} \\ 
                    Dynamic smoke 3
                }
                \label{fig:dynamic_3}
            \end{subfigure}
%%%%%%%%%%%%%%%%%%%%%%%%%%%%%%%%% second ROW
            \begin{subfigure}[b]{0.3\textwidth}     % eindritteln der Textfläche für neben einander
                \centering
                \includegraphics[width=\textwidth]{pics/ownData/dynamicSmoke_s1.jpg}
                \caption{\textbf{dynamic Smoke}  \\
                    Pic 1
                }
                \label{fig:dynamic_1}
            \end{subfigure}
            ~       % when I am Right, this should mean that they are placed next to each other
            \begin{subfigure}[b]{0.3\textwidth}
                \centering
                \includegraphics[width=\textwidth]{pics/ownData/rotation_s1.jpg}
                \caption{\textbf{more dynamicsmoke} \\ 
                    Picture 2
                }
                \label{fig:dynamic_2}
            \end{subfigure}
            ~
            \begin{subfigure}[b]{0.3\textwidth}
                \centering
                \includegraphics[width=\textwidth]{pics/ownData/stain_s1.jpg}
                \caption{\textbf{even more dynamic smoke} \\ 
                    Dynamic smoke 2
                }
                \label{fig:dynamic_3}
            \end{subfigure}
%%%%%%%%%%%%%%%%%%%%%%%%%%%%%%%%% third ROW
            \begin{subfigure}[b]{0.3\textwidth}     % eindritteln der Textfläche für neben einander
                \centering
                \includegraphics[width=\textwidth]{pics/ownData/dynamicSmoke_s2.jpg}
                \caption{\textbf{dynamic Smoke}  \\
                    Pic 1
                }
                \label{fig:dynamic_1}
            \end{subfigure}
            ~       % when I am Right, this should mean that they are placed next to each other
            \begin{subfigure}[b]{0.3\textwidth}
                \centering
                \includegraphics[width=\textwidth]{pics/ownData/rotation_s2.jpg}
                \caption{\textbf{more dynamicsmoke} \\ 
                    Picture 2
                }
                \label{fig:dynamic_2}
            \end{subfigure}
            ~           
            \begin{subfigure}[b]{0.3\textwidth}
                \centering
                \includegraphics[width=\textwidth]{pics/ownData/stain_s2.jpg}
                \caption{\textbf{even more dynamic smoke} \\ 
                    Pictu 3
                }
                \label{fig:dynamic_3}
            \end{subfigure}
            ~
            \caption{Results of the Data Acquirement}
            \label{fig:DataAcq}
        \end{figure*}
        % This somehow does not display as expected....
        % maybe add even more pictures to fill the page.... I mean: there are more then enough...

        \newpage

	\subsection{Real Data}
        \cmt{Last but not least} \\
        \cmt{Creds to Uwe Widmair} \\
        \newline
        \cmt{Pictures out of his different setups}
        \newpage

        Also put pictures relevent to UWE here.
        \newpage
%%%%%%%%%%%%%%%%%%%%%%%%%%%%%%%%%%%%%%%%%%%%%%%%%%%%%%%%%%%%%%%%%%%%%%%%%%%%%%%
%% MATH
%%%%%%%%%%%%%%%%%%%%%%%%%%%%%%%%%%%%%%%%%%%%%%%%%%%%%%%%%%%%%%%%%%%%%%%%%%%%%%%
\section{Algorithmic \& Mathematical}

	\subsection{Fundamentals}
		\cmt{explanations about.... Color representation? - RGB BGR LAB}
        \fuck{Bring this into own Main Chapters!}

	\subsection{Detection}
		\subsubsection{Riot Detection in CCTV}
        \subsubsection{\cmt{another one would be nice}}
	\subsection{Filters}
		\subsubsection{Dark Channel Prior}
		\subsubsection{Model Based Removal}
		\subsubsection{Chromaticity Based Removal}
        \subsubsection{Depth Estimates}
		
	\subsection{Combination}
		\cmt{first find - then filter}
	
	\subsection{Implementation}
		The selected framework for the video manipulation is openCV \cmt{REF}. Used via the Pyhton3.6.7 on a Linux 18.04 with a 4.15.0-43 generic Kernel.\\
	
%%%%%%%%%%%%%%%%%%%%%%%%%%%%%%%%%%%%%%%%%%%%%%%%%%%%%%%%%%%%%%%%%%%%%%%%%%%%%%%
%% CODE
%%%%%%%%%%%%%%%%%%%%%%%%%%%%%%%%%%%%%%%%%%%%%%%%%%%%%%%%%%%%%%%%%%%%%%%%%%%%%%%
\section{Code}
	\subsection{Mathematical functions - Implementation}

    	\begin{lstlisting}[language=Python]
def testingMode(self, frame):
    #create contrast
    mask = []
    mask.append(cv2.cvtColor(frame, cv2.COLOR_BGR2LAB))
    l, a, b = cv2.split(mask[0])

    #clahe means "Contrast Limited Adaptive Histogram Equalization"
    clahe = cv2.createCLAHE(clipLimit=3.0, tileGridSize=(4, 4))
    cl = clahe.apply(l)		# reduce contrast

    limg = cv2.merge((cl, a, b))
    mask.append(limg)

    res = cv2.cvtColor(limg, cv2.COLOR_LAB2BGR)

    return res, mask

		\end{lstlisting}

