\section{Data Acquisition}
	A early encountered problem is the availability of a database of smoked and noisy images or videos that are essential for the development of the algorithms. \\
	The need of a video database showing static and dynamic environments as well as movements and different kinds of noise combinations ended in the decision to create laparoscopic testing simulator.
	This simulator enabled the recording of videos a surgical like setup. \\

	Additional to the generation of on simulated test data, real surgery data has been obtained from the Kreisklinken Günzburg-Krumbach. \\
	While the data created within the simulator is used to develop detection and removal algorithms, the real surgical videos are only used to examine the algorithms. This is done in order to avoid over fitting the algorithms to special videos.\\

	\subsection{Endoscopic Video Data}
		The objective is to create data in a custom setup, outside of the body of any creature. The simulation consists of a capturing setup and multiple iterations of data grabbing, with different documented changes to the environment as well as to the cauterized objects.\\

	The archived data will contain a set of \textbf{videos} and \textbf{images} that show \textit{noise}. This noise is defined by:\\
		\paragraph{surgical smoke \\} 
			Surgical smoke occurs when surgeons cauterize organs, veins or other body internal objects.
				Although its contents are mostly steam, there are multiple other particles dispensed within this kind of Fog. The applied thermal energy of the cauter causes the tissues cells to explode. The exploding cell therefore releases its contents into the environment. This smoke is mostly composed by steam as well as tissue particles, but can also contain tentative bacteria \cite{contentOfSmoke} \\
		Depending on the time, type and tissue of the cauterization the smoke can vary in its aspect, also the carbonisation level and the kind of tissue influences the smoke and its aspect. \\
        This surgical smoke is seen in picture\ref{fig:realNoise} under the point \textbf{(1)} in the upper right area.

		\paragraph{haze \\} 
            Haze is a effect that also appears alongside to cauterization, the heat of the surgical cauters leads to the vaporization of body liquids. This leads to fog, steam and haze that can block the direct vision to the targeted areas and also can build a thin film on the endoscopes objective, as well as \textit{staining} which shows up as the formation of drops on the objectives. An example of haze is given in point \textbf{(2)} in picture\ref{fig:realNoise}.

		\paragraph{staining \\} 
			Multiple body internal fluids get directly in touch with the endoscopes objective. These fluids gather to build drops, which blur or cover certain parts of the cameras vision. 
            
            \begin{figure}[h]
                \centering
                \includegraphics[scale=.35]{pics/schmauch/noisesMarked.png}
                \caption{Noises in real Laparoscopic surgeries \\ 
                    \textbf{(1)}surgical smoke, \textbf{(2)}haze caused by steam \\
                    \cmt{PIC CANNOT BE USED - LEGAL ISSUES}
                }
                \label{fig:realNoise}
            \end{figure}

	\subsection{Video Data Requirements}
	All of the above mentioned effects are required to occur withing the generated test data, as it would be the case in a real surgery. 
	The Quality of the images should be quite high, since typically used endoscopes in modern surgery often have high frame rates (above 60FPS) and resolutions of more than 4K2K (4000x2000 Pixels).
	\\
	The produced data focuses on a setup that is as close as possible to a real surgery. \\
	The quality of the images is therefore just a secondary goal. More important is the creation of videos, as well as images that contain multiple combinations of the given noises. \\
    
    A special requirement for on the data is a setup in which the noise slowly rises, which might not be a real world situation but helps to implement code and debug the algorithms. \\
    Following, the self generated data need to contain videos with fixed movements and planned noises. Their generation is based on laparoscopic video data, from which the most common movements and noise combinations will be derived and imitated.\\

\section{Algorithms}
    
    Popular literature focuses on the removal of surgical smoke within the whole image. Two new aspects will be introduces in this paper.\\
    \newline
    The first one is to find noises within the given video-files and the second one will be the removal of not only surgical smoke, but also fog, haze and staining. \\
    The removal of the noises is then performed only in the certain sections of the videos frames, where they actually occur. Instead of the common approach, which applies filters on the whole frame.\\
    This aims to archive a selective filter, with can find different kinds of noises to then apply the proper filter to remove them within a single frame.
    So every noise will be treated with the corresponding algorithm to remove it. \\
    \newline

    Based on picture\ref{fig:realNoise} the noisy areas could be detected as shown in image\ref{fig:coloredNoise} and different filters can be applied for both the surgical smoke \textbf{(1)} and haze \textbf{(2)}.

    \begin{figure}[h]
        \centering
        \includegraphics[scale=.35]{pics/schmauch/coloredNoises.png}
        \caption{Areas to apply filters on.\\
            \textbf{(1)}smoke filters, \textbf{(2)}haze removal
            \cmt{PIC CANNOT BE USED - LEGAL ISSUES}
        }
        \label{fig:coloredNoise}
    \end{figure}
    
\subsection{Detection}
    The detection of noises is fundamental for a succeeding in the implementation of the filters, only a reliable detection of noises will ensure that the proper filters are selected and applied to the given area of the image.

    \paragraph{Smoke Detection \\}
        A widespread application for smoke detection is within traffic- and riot-control \cite{vision}. Mostly these algorithms rely on smoke diffusion. In order to detect smoking objects they compare following frames and detect Smoke on its dispersal behaviour.\\
    
    \paragraph{Haze Detection \\}
        The detection of haze, respectively steam is a difficult task, but a very common problem within the surgical setup, since almost all cutting or sealing procedures are performed using a cauter, which heats up tissue and liquids.\\
        
    \paragraph{Stain Detection \\}
        Staining can be caused by body internal fluids like blood, lipids or lymph. These fluids tend to sputter into the surrounding as they are heated rapidly. \\
        When these kinds of pollutions occur on the image the surgeons have to remove the endoscope from the patients body. This procure costs often critical time.

\subsection{Removal}
    To remove the previously detected noises different Filters will be evaluated for their suitability. \\
    
    \paragraph{Contrast Based}
        Contrast - works with Smoke - \cite{detectSmoke} and \cite{imageDehaze},
    \paragraph{Dark Channel prior}
        Dark Channel Prior,
    \paragraph{Bayesian Inference}
        Bayesian Inference, \cite{jointDesmoke}

%%%%%%%%%%%%%%%%%%%%%%%%%%%%%%%%%%%%%%%%%%%%%%%%%%%%%%%%%%%%%%%%%%%%%%%%%%%%%%%
%%%% Taken from Methodic up 
%%%%%%%%%%%%%%%%%%%%%%%%%%%%%%%%%%%%%%%%%%%%%%%%%%%%%%%%%%%%%%%%%%%%%%%%%%%%%%%
\fuck{Check this section again with ANE-comments}
\section{Bench Test Background}
	Surgical smoke is different to ''normal'' smoke because of its contents and their composition as given by the paper \cite{contentOfSmoke}. Therefore the generation of the test data needs to be performed with a special focus on the realism of the smoke and haze. Also not only the cauterized organs and their humidity are important to the aspect of the smoke. According to experts also the method as well as the duration of cauterization is an important influence on the smokes quality.  \newline

	Training situation for doctors-to-be include exercises in laparoscopic simulations, so called phantoms. These phantoms are a emulation of a real laparoscopic surgery and 
	The paper \cite{MIS_host} introduces a learning method for expectant surgeons to practice without living patients, be it human or animal. Since these surgeries are expensive, time consuming and moreover they need to be authorized by governmental or veterinarian institutions in the European Union. \newline

	The data generation of surgical smoke and artefacts requires the use of organic material that can be cauterized. Porcine visceral organs are very similar to human visceral organs which the paper \cite{swineModel} postulates and are therefore a appropriate candidate for the test setups. \newline

	Laparoscopic trainers are often used in eduction and training of future surgeons. Therefor a source for the setup of these kind of trainers and how they should look like is taken from the papers \cite{MIS_host} and \cite{MIS_trainer}. \newline

%%%%%%%%%%%%%%%%%%%%%%%%%%%%%%%%%%%%%%%%%%%%%%%%%%%%%%%%%%%%%%%%%%%%%%%%%%%%%%%
%% PREQUISITES
%%%%%%%%%%%%%%%%%%%%%%%%%%%%%%%%%%%%%%%%%%%%%%%%%%%%%%%%%%%%%%%%%%%%%%%%%%%%%%%
\section{Test Perquisites}
		In order to capture videos that are related to real surgeries multiple factors need to be considered. Since not all animals have tissue similar to humans a suitable host needs to be selected.\\

	\paragraph{\textbf{Mice} \textit{(Mus musculus)}\\} have a quite high similarity to the human body, but problems occur in the availability and the handling of these organs. Therefore Mice's organs drop out, since they are expected to have a bad handling due to their small size. \newline
	\newline
	\paragraph{\textbf{Porcine Organs}\textit{(Sus scrofa domesticus)}\\} fit better for this purpose, but especially their internals are not very commonly consumed. For closer testings it is expected that porcine organs are necessary, while for first setup tests or technical evaluations other organs might be sufficient.\newline
	\newline
	\paragraph{\textbf{Bovine Organs} \textit{(Bos primigenius taurus)}\\} are the most commonly available visceral organs, since they are often used in culinary way. Tripes or livers especially. Although their similarity to human internals is not that high, their advantages in availability and usability makes them a good choice for first experiments.\newline
	\newline	 \\
	As given in \cite{swineModel}, the common domestic pig has already often been used as the major species for animal testing of both pharmaceutical and surgical experiments. Due to the fact that pigs have a lot of similarities to the human body, e.g. the resemblance of their cardiovascular systems or the digestive systems.\newline

	Concluding to this, the data should contain a set of video footage that is based on the cauterization of porcine internals.

	\paragraph{Organs\\}
    	The most laparoscopic surgeries are performed in the area of the abdomen, so the most reasonable things to use as target objects for the cauterization would include:
		\begin{itemize}
			\item Gallbladders ( Vesica biliaris),
			\item Livers ( Hepar ),
			\item Stomaches ( Gaster ),
			\item Gastrointestinal tracts ( Tractus digestorius ).
		\end{itemize}

		The selection of organs depends on similar choices as the selection of animals. The decision is based on their availability and usability, countering to real surgeries like liver resections or cholecystectomies.\newline
		Another point is the internal moisture that is given by mucous membranes, their cauterisation leads to fog and steam. This issue needs to be considered since moisture is expected to create foggier images that therefore are closer to real surgeries within wet environments.\\
\newline
		This moisture occurs in forms like Blood, lipid and other body internal fluids. By heating up these liquids different forms of steam, fog and stains are created which all have different aspects. Based on this information a insertion of Blood, etc. into the simulation needs to be considered. \\ 
	\newline
       Body internal humidity has impact on the distribution of smoke and staining within the surgical area (e.g. a \gls{Pneumoperitoneum}). 

	\paragraph{Hardware \\}
        The hardware setup includes:\\
        \cmt{Table to big - fix required}
        \begin{table}[H]

            \centering
            \label{tab:Cameras}
            \begin{tabular}{l|l|l|l}	%all left aligned
                \textbf{Company} & \textbf{Model} & \textbf{Type} & \textbf{REF}\\ 
                \hline
                Wolf & Endocam & Bronchoscope & \cite{endocam} \\		%1CCD Endocam 5520 
                Wolf & 3D-Endocam & Stereoendscope & \cite{} \\		%3D-HD-Endocam F1014 
                IDS & ui-3240cp & 1280x1024/60FPS& \cite{} \\ 
                Nikon & D3400 & 6000x4000 & \cite{dslr} \\
                Logitec & C270 & 1280x720 & \cite{webcam} \\
                Alcor Micro & SJ00446 & 1280x170 & \fuck{NONE} \\
                \hline
                Epiphan & DVI2USB3.0 & 1920x1080/30FPS & \cite{epiphan} \\
                \hline
                Aesculap & Caiman 5 & Vessel sealer & \cite{sealer} \\
                Aesculap & Letrafuse & RF-Generator & \cite{hfgen} \\
            \end{tabular}
            \caption{Capturing Hardware}
        \end{table}

    The Cameras are inserted into a self-made laparoscopic trainer\ref{fig:phantomI}.
        
    \begin{figure}[H]
        \centering
        \includegraphics[scale=0.06]{pics/setups/Laparoscope_box.jpg}
        \caption{self build Laparoscope phantom}
        \label{fig:phantomI}
    \end{figure}
    
    With penetrable neoprene cloth to simulate skin. \ref{fig:phantomII}.

    \begin{figure}[H]
        \centering
        \includegraphics[scale=0.06]{pics/setups/Laparoscope_box_cover.jpg}
        \caption{neoprene cloth with holes form tool insertion}
        \label{fig:phantomII}
    \end{figure}
	
	\subsection{Required Data}
		
	The resulting data needs to fulfill several requirements, mostly concerning the occurrence and mixture of noise the as well as the quality of the images.\newline
	Investigated noises are: \\

	\begin{itemize}
		\item Surgical Smoke,
		\item Haze,
		\item Staining of
		\begin{itemize}
			\item Blood,
			\item Water,
			\item Lipids.
		\end{itemize}
	\end{itemize}
	
    The generation of data is based on video data given by the \gls{kkhgzkru} \fuck{REF broken?}.
	These videos have been used as a template to generate own Data corresponding to the given real videos.
    The videos show prevalent movements of surgeries which are reproduced as close as the setup and experience allow.\\
    \\
    As the generated Data will used as base for the code generation additional requirements besides the realism aspect need to be met.\\ 
    One of these conditions is the slow rise of noise, which is expected to support the algorithm creation. By e.g. showing the thresholds for recognition or reconstruction.
