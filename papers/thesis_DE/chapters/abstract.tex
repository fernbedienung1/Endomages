Das Ziel der vorliegenden Thesis ist die Verbesserung minimal invasiver Eingriffe durch eine Verbesserung der laparoskopischen Bildgebung.
Konkret behandelt dieses Paper die Erkennung und Entfernung von \gls{surgical smoke} sowie das Beschlagen der Kamera oder andere Verunreinigungen, die durch Werkzeuge des chirurgischen Eingriffes entstehen können.
Dieser sogenannte \gls{surgical smoke} vehindert eine klare Sicht auf die Operationsgebiete, dies verzögert und verkompliziert die Eingriffe.\\
\newline
Um Algorithmen implementieren zu können werden zuerst Anforderungen an die Videodaten definiert, um folgend eine Methode vorzustellen die eigene laparoskopische Videodaten erzeugen kann. Diese Daten werden evaluiert, diskutiert und aufbereitet.\\
Mathematische Grundlagen der Bildbearbeitung werden entsprechend ihres Nutzens für die Schmutzentfernung betrachtet und implementiert.\\
\newline
Abschließend werden die Ergebnisdaten präsentiert und weiterführende Arbeit vorgeschlagen.

%Um diesen Effekten entgegenzuwirken wird eine Methode vorgestellt die es ermöglicht Verunreinigungen zu erkennen, sie zu unterscheiden und entsprechende Filter anzuwenden.
%Die Anwendung der Filter erfolgt passend zur erkannten Verunreinigungen nur in dem Bereich in dem diese auch auftritt, um zu gewährleisten, dass das Bild in unverschmutzten Regionen nicht verändert und so verfälscht wird.\\
\paragraph{Glossary}
%\printglossaries

\paragraph{Keywords}
    endoskopische Bildgebung, Chirurgie, Dampf, Rauch, surgical smoke, Raucherkennung
