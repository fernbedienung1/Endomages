Diese Arbeit beschreibt eine Methode um \textit{Surgical Noise} aus Bildern und Videos laparoskopischer Eingriffe zu entfernen.
Dieses Entfernen geschieht vollständing mittels Software, es wird keine zusätzliche Hardware benötigt um eine Verbesserung herbeizuführen.
Die Idee dieser Arbeit basiert auf einer Hospitation einer laparoskopischen Cholezystektomie vom 18.05.2018 im \gls{ADK}. \\
Eine Befragung der Chirurgen ergab dass Schmutz, Rauch und Dampf im Bild des Endoskopes die Arbeit deutlich behindern und somit den Arbeitsfluss der Chirurgen stören.

\section{Hintergrund}

    Laparoskopische Eingriffe erfordern spezielle Werkzeuge, wie beispielsweise Kauter, welche durch ihre Verwendung Verschmutzungen erzeugen. Durch das \gls{Pneumoperitoneum} können Verschmutzungen des Endoskopes, wie beispielsweise Rauch und Dampf, das Operationsgebiet nicht verlassen und behindern dadurch das Sichtfeld der Endoskopekamera.\\
    Eine reine Softwarelösung des Problems ist erstrebenswert um Anschaffungskosten und zeitlichen Aufwand während der Operation gering zu halten.

\iffalse
    \begin{figure}[h]
        \centering
        \includegraphics[width=\textwidth]{pics/schmauch/noisesMarked.png}
        \caption{
           Verschiedene Arten von ''Surgical Smoke''
        }
        \label{fig:introNoise}
    \end{figure}

    Bild \ref{fig:realNoise} zeigt einen Ausschnitt einer Operation bei der \textit{Surgical Smoke} und Beschlagen der Kamera die Sicht auf das Operationsgebiet behindern.\\
\fi

\section{Ziel}
    Das Ziel dieser Arbeit ist es deshalb diesen \textit{Surgical Noise} Softwareseitig mittels Bildbearbeitungsalgorithmen zu erkennen und zu entfernen.\\
    Während der Bearbeitung des Themas wurde es notwendig eine entsprechende Datenbasis, die Aufnahmen von \textit{Surgical Noises} enthalten, selbst zu generieren. Diese Daten sind die Grundlage der Erkennungs- und Entfernungsalgorithmen die im Rahmen der Arbeit entstehen sollen.\\

    Um dieses Ziel zu erreichen wurden folgende Unterpunkte erstellt welche das Hauptproblem aufteilen und Teilprobleme definieren:\\
    \begin{enumerate}
        \item Generation von Bildern die Störungen enthalten,
        \item Erkennen der Störungen,
        \item Entfernen der Störungen,
        \item Evaluation und Vergleich der generierten Bilder mit realen intraoperativen Bildern.
        \item Evaluation der bereinigten Bilder im Vergleich zu den unbereinigten Bildern
    \end{enumerate}

    Folglich soll das Endergebnis Störungen in Bildern und Videos finden und entsprechend der gefundenen Störung einen passenden Filter auf den Bildbereich anwenden der die Störung enthält. \\
    \newline
    Das daraus resultierende Bild soll eine verbesserte Sicht auf das Operationsgebiet zeigen, im Vergleich zum ursprünglichen Eingangsbild der Laparoskop Kamera.\\
    Im Rahmen der Möglichkeiten dieser Arbeit wird eine Methode vorgestellt entsprechendes Bildmaterial zu generieren, zu bearbeiten und zu evaluieren. \\
    Die generierten Daten werden mit realem intraoperativen Videomaterial verglichen und bereits bekannte Bildbearbeitungsalgorithmen werden implementiert um eine Verwendung zu demonstrieren.\\
