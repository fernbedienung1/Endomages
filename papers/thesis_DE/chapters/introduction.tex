Diese Arbeit beschreibt eine Mehtode um \textit{Surgical Noise} aus Bildern und Videos laparoscopischer Eingriffe zu entfernen. \\
Dieses entfernen geschiet vollständing mittels software, es wird keine zusätzliche Hardware benötigt um eine Verbesserung herbeizuführen. \\
\newline
Die Idee dieser Arbeit basiert auf einer Hostpitaion einer Laparoscopischen Cholezystektomie vom XX.XX.XXXX im \gls{ADK}. \\
Eine Befragung der Chirurgen ergab dass Schmutz, Rauch und Dampf im Bild des Endoskopes die Arbeit deutlich behindern und somit dem Arbeitsfluss der Chirurgen stören.

\section{Hintergrund}
    
    Laparoskopische Eingriffe erfordern spezielle Werkzeuge, wie beispielsweise Kauter (Rechts im Bild \ref{fig:introNoise}), die durch ihre Verwendung Verschmutzungen erzeugen. Durch das notwendige \gls{Pneumoperitoneum} können Verschmutzungen wie beispielsweise Rauch und Dampf das Operationsgebiet nicht verlassen und behindern dadurch das Sichtfeld der Endoskope Kamera.\\
    Eine reine Software Lösung des Problems ist erstrebenswert um Anschaffungskosten und zeitlichen Aufwand während der Operationen gering zu halten.
    
    \begin{figure}[h]
        \centering
        \includegraphics[width=0.5\textwidth]{pics/schmauch/realNoise.jpg}
        \caption{
           Verschiedene Arten von ''Surgical Smoke'' 
        }
        \label{fig:introNoise}
    \end{figure}

    Bild \ref{fig:introNoise} zeit einen Ausschnitt einer Operation bei der \textit{Surgical Smoke} und Beschlagen der Kamera die Sicht auf das Operationsgebiet behindern.\\
    

\section{Ziel}
    Das Ziel dieser Arbeit ist es deshalb diesen \textit{Surgical Noise} Software seitig mittels Bildbearbeitungsalgorithmen zu erkennen und zu entfernen.\\
    Während der Bearbeitung des Themas wurde es notwendige eine entsprechende Datenbasis, die mit Aufnahmen von \textit{Surgical Noises} enthalten selbst zu generieren. Diese Daten sind die Grundlage der Erkennungs- und Entfernungs- Algorithmen die im Rahmen der Arbeit entstehen sollen.\\
    \newline

    Um dieses Ziel zu erreichen wurden folgende Unterpunkte erstellt welche das Hauptproblem aufteilen und Teilprobleme definieren:\\
    \begin{enumerate}
        \item Generation von Bildern die Störungen enthalten,
        \item Erkennen der Störungen,
        \item Entfernen der Störungen,
        \item Evaluation und vergleich der Generireten Bilder mit realen intraoperativen Bildern.
        \item Evaluation der Bereinigten Bilder im vergleich zu den unbereinigten Bildern
    \end{enumerate}

    Folglich soll das Endergebnis Störungen in Bildern und Videos finden und entsprechend der gefundenen Störung einen passenden Filter auf den Bildbereich anwenden der die Störung enthält. \\
    \newline
    Das daraus resultierende Bild soll eine verbesserte sicht auf das Operationsgebiet zeigen, im vergleich zum ursprünglichen eingangesbild der Laparoskop Kamera.
