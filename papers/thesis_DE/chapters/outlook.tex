\section{Aktueller Stand}
    Mit den Generierten Daten wurde bis jetzt mit Hilfe von Octave und Python begonnen Algorithmen zu implementieren die das Gesamtbild manipulieren.

\section{nächste schritte}


\section{Erkennen der Verschmutzungen}
    Ein aktuelles Thema sind Neuronale netze und deren Fähigkeit auf bestimmte Muster, insbesondere in der Bilderkennung eingelernt werden zu können.
    Durch die Hilfe solcher Neuronaler Netze wäre es gegebenenfalls möglich eine Intelligente Erkennung der Verschmutzungen zu implementieren, als Grundlage dafür können die im Rahmen dieser Thesis entstandenen Videodaten und Bilder verwendet werden. Oder auf deren Grundlage weitere erstellt werden.\\

\section{Entfernung in Bereichen}
    Wie bereits im Kapitel Diskussion \ref{sec:Discussion} angemerkt wurde, wird durch die Bearbeitung der Bilder die Realität verzerrt dargestellt.
    Ein interessanter Punkt wäre dementsprechend die Verwendung der Algorithmen nur in Bestimmten Bereichen des Bildes.\\
    \newline

    In sinnvoller Kombination mit einer erkennen der Verschmutzungen wäre es so möglich nur die Teile der Bilder zu bearbeiten die tatsächlich verschmutzt sind.

