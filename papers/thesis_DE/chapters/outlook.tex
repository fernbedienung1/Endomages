

    Die generierten Bilder sollten im Laufe der Arbeit noch Chirurgen zur Validierung vorgelegt werden, aufgrund zeitlich nicht kompatibler Verfügbarkeiten der Chirurgen ist dies jedoch nicht mehr im Rahmen der Arbeit zustande gekommen. \\
    Die generierten Daten bedürfen daher noch einer Prüfung durch Mediziner, Chirurgen oder andwereitig qualifizierte Experten, die bestätigen oder verneinen dass es sich hierbei um realitätsnahe Daten handelt.\\
    Ausserdem ist zu klären unter welchen Aspekten sich die generierten Daten von realen Daten unterscheiden und in wiefern dies eine Weiterverarbeitung beeinflusst.

\section{Aktueller Stand}
    Mit den generierten Daten wurde bis jetzt mit Hilfe von Octave und Python begonnen Algorithmen zu implementieren die das Gesamtbild manipulieren.

\section{Nächste Schritte}


\section{Erkennen der Verschmutzungen}
    Ein aktuelles Thema sind neuronale Netze und deren Fähigkeit auf bestimmte Muster, insbesondere in der Bilderkennung, eingelernt werden zu können.
    Durch die Hilfe solcher neuronaler Netze wäre es gegebenenfalls möglich eine intelligente Erkennung der Verschmutzungen zu implementieren, als Grundlage dafür können die im Rahmen dieser Thesis entstandenen Videodaten und Bilder verwendet werden. Oder auf deren Grundlage weitere erstellt werden.\\

\section{Entfernung in Bereichen}
    Wie bereits im Kapitel Diskussion \ref{sec:Discussion} angemerkt wurde, wird durch die Bearbeitung der Bilder die Realität verzerrt dargestellt.
    Ein interessanter Punkt wäre dementsprechend die Verwendung der Algorithmen nur in bestimmten Bereichen des Bildes.\\
    \newline

    In sinnvoller Kombination mit einem Erkennen der Verschmutzungen wäre es so möglich nur die Teile der Bilder zu bearbeiten die tatsächlich verschmutzt sind.
