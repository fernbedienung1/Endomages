\section{Generierung von Basisdaten}
    Durch den Mangel an Videomaterial dass die im Grundlagenkapitel \ref{sec:Fundamentals} beschriebenen Verunreinigungen enthalten ist es notwendig diese Daten selbst zu generieren.


\subsection{Laparoskopische Laborversuche}
    Während der Bearbeitung des Themas war ein schnell erkanntes Problem dass nur weniger der beschrieben Videos existieren. Deshalb wurde es notwendig eigene Daten zu generieren, denn insbesondere verrauchte und verschmutzte Bilder werden vom Chirurgen entsorgt, da diese die für Mediziner notwendigen Areale verdecken.
    Zusätzlich dazu spielen rechtliche Aspekte eine Rolle, denn Patientendaten dürfen nur mit ausdrücklichem Einverständnis des betreffenden Patienten veröffentlicht werden. \\
    Um die benötigten Daten selbst zu generieren wurden 2 Versuchsreihen erstellt.\\
    Prototyp 1 sollte mit relativ wenig Aufwand bereits brauchbare Ergebnisse und Videoaufnahmen von Rauch und Dampf erzeugen.
    Nach einem Fehlschlag dieses Vorhabens wurde unter deutlichem Mehraufwand ein zweiter Laborversuch initiiert, welcher brauchbare Ergebnisse lieferte.

    \clearpage
    \subsubsection{Erster Prototyp}
        Der erste Prototyp besteht aus leicht zugänglichen Materialien und sollte das Problem der fehlenden endoskopischen Bilder mit wenig Aufwand lösen.\\
        Zum Einsatz kamen die in Tabelle\ref{tab:usedHW_I} verwendenden Geräte.\\
        Hauptaugenmerk des ersten Prototypes war eine ''Machbarkeitsüberprüfung'', bei der das Ziel war herauszufinden ob mittels der verwendeten Kameras Bilder erzeugt werden können, die Laparoskopien ähnlich sind, und sich als Basis für die Algorithmenentwicklung eignen.
        \newline

        \begin{table}[H]
            \centering
            \begin{tabular}{l|l|l|l}	%all left aligned
                \textbf{Hersteller} & \textbf{Model} & \textbf{Type} & \textbf{REF}\\
                \hline
                Nikon & D3400 & 6000x4000 & \cite{dslr} \\
                Logitec & C270 & 1280x720 & \cite{webcam} \\
                Alcor Micro & SJ00446 & 1280x170 & \cite{light}\\
                \hline
                Weller & Lötkolben & std & \cite{iron} \\
                \hline
                Selfmade & V01 & Umgebungssensor & \cite{RJ} \\
            \end{tabular}
            \caption{Aufnahme Geräte, Kauter und Sensoren\label{tab:recHW_1}}
        \end{table}

        Tabelle \ref{tab:recHW_1} listet die die verwendeten Komponenten für die Vorabversuche.
        Die Geräte sind insbesondere für Laien erschwingliche und verfügbare Geräte. Ziel des Versuchs ist es einen Gesamteindruck über die Komplexität der Versuche zu erhalten, um die Komplexität selbst sowie den Aufwand für Folgeversuche abzuschätzen.

        \begin{table}[H]
            \centering
            \begin{tabular}{c|l|l}
                \textbf{REF} & \textbf{Gerät} & \textbf{Beschreibung} \\
                \hline
                \textbf {1} & Raspberry PI & Steuergerät\\
                \textbf {2} & DSLR & Haupt-Kamera\\
                \textbf {3} & Webcam & USB-Kamera\\
                \textbf {4} & Endoscope & Amateur Endoskope\\
                \textbf {5} & Umgebungssensor & Temperatur/Feuchtigkeit\\
                \textbf {6} & Lötkolben & improvisierter Kauter\\
                \textbf {7} & LED licht & Lichtquelle innerhalb des Trainers\\
            \end{tabular}
            \caption{Komponenten des ersten Prototypen, Nummern referenzieren \ref{fig:MK_I_schematic} und \ref{fig:MK_I_real}\label{tab:usedHW_I}}
        \end{table}

%%%%%%%%%%%%%%%%%%%%%%% Pictures of Test Setup %%%%%%%%%%%%%%%%%%%%%%%%%%%%%%%%%%

	    \begin{figure*}[hp]
             \begin{subfigure}[b]{\textwidth}
                 \centering
                 \includegraphics[scale=0.3]{pics/setups/setup1_schematic.png}
                 \caption{\textbf{Schematischer Aufbau} \\
                     (1)Raspberry PI B3+, (2)DSLR, (3)Webcam, (4)Endocam, (5)Temperatur- und  Luftfeuchtigkeits-Sensor, (7)LED-Licht \\
                     \textit{Geräte wie in Tabelle \ref{tab:usedHW_I}.} \\
                 }
                 \label{fig:MK_I_schematic}
             \end{subfigure}

             \hspace{2cm}

             \begin{subfigure}[b]{\textwidth}
                 \centering
                 \includegraphics[scale=0.3]{pics/setups/setup1_real.png}
                 \caption{\textbf{Realer Aufbau} \\
                     (1)Raspberry PI B3+, (2)DSLR, (3)Webcam, (4)Endocam, (5)Temperatur- und  Luftfeuchtigkeits-Sensor, (7)LED-Licht \\
                     \textit{Geräte wie in Tabelle \ref{tab:usedHW_I}.}
                 }
                 \label{fig:MK_I_real}
             \end{subfigure}

             \hspace{2cm}

             \begin{subfigure}[b]{\textwidth}
                 \centering
                 \includegraphics[scale=0.15]{pics/setups/setup1_result.png}
                 \caption{\textbf{Ergebnis Bilder} \\
                     Bilder aus den Videos des Aufbaus\ref{fig:MK_I_real}. \\
                     (1)DSLR-output, (2)Webcam-output, (3)Endocam-output.
                 }
                 \label{fig:MK_I_result}
             \end{subfigure}

             \caption{Prototype für Laborversuche}
             \label{fig:MK_I}
         \end{figure*}
        \newpage

    \paragraph{Fazit des ersten Prototpen}
        Die Generierung von Daten ist mit ''Haushaltsmittlen'' nicht zu ermöglichen. Zwar ist es möglich Rauch zu erzeugen, die ersten Tests und Versuche decken einige Fallstricke auf, diese jedoch nicht in einem realen oder brauchbaren Ausmaß. Für die brauchbare Verwendung für die Algorithmen muss mit professionelleren Werkzeugen gearbeitet werden. Die Verbesserungen sind vor allem in den Bereichen: \\
        \begin{itemize}
            \item Laparoskopischer Trainer,
            \item Qualität der Versuchsorgane,
            \hspace{1cm}
            \item \textbf{Endoskope} und
            \item \textbf{Kautern}.
        \end{itemize}

        gemacht worden.\\
        Folglich ist es notwendig einen zweiten Laborversuch durchzuführen welcher die obrigen Punkte berücksichtigt.

    \subsubsection{Zweiter Prototyp}

        Aus den Ergebnissen des ersten Prototypes ging hervor, dass es ohne echte Endoskope und Kauter nur schwer möglich ist Daten zu erzeugen die realen Operationen ähneln. Als Resultat dessen sind die eingesetzten Kameras, Werkzeuge und auch der Trainer, in dem die Versuche durchgeführt wurden, verbessert worden.. \\

    \begin{table}[h]
        \centering
        \begin{tabular}{|l|c|}
            \hline
            \textbf{Prototyp 1} & \textbf{Prototyp 2 Verbesserung} \\
            \hline
            Organe & Frische Organe (Schlachtung am Versuchstag (21.12.2018)\\
            Phantom & angepasst an laparoskopische Trainer \\
            Endoskope & Leihgaben des DKFZ und Prof.Dr. Alfred Franz \\
            Kauter & Kauter und HF-generator, Leihgaben der NewTec GmbH. \\
            \hline
        \end{tabular}
        \caption{ Verbesserungen der Testumgegbung \label{tab:improvements}}
    \end{table}

        Erwartungsgemäß können mit passenderen Organene (Schwein, frische, etc. \ref{sec:Fundamentals} Bessere Ergebnisse erwartet werden. Deshalb sind die folgenden Versuche mit frischen Organen durchgeführt worden. Zusätzlich zu den professionellen Werkzeugen\ref{tab:recHW_2} wird so eine Verbesserung zum ersten versuch erwartet. \\

        \begin{table}[H]
            \centering
            \begin{tabular}{l|l|l|l}	%all left aligned
                \textbf{Hersteller} & \textbf{Model} & \textbf{Type} & \textbf{REF}\\
                \hline
                Wolf & Endocam & Bronchoscope & \cite{endocam} \\
                Wolf & 3D-Endocam & Stereoendscope & \cite{3DHD} \\
                IDS & ui-3240cp & 1280x1024/60FPS& \cite{ids} \\
                \hline
                OSRAM & Xenophot & 150W Halogen & NONE \\
                \hline
                Aesculap & Caiman 5 & Vessel sealer & \cite{sealer} \\
                Aesculap & Letrafuse & RF-Generator & \cite{hfgen} \\
            \end{tabular}
            \caption{Aufnahmegeräte und Kauter\label{tab:recHW_2}}
        \end{table}

        Der Aufbau des 21.12.2018 beinhaltet die Hardware aus Tabelle \ref{tab:recHW_2}.\\
        Zweiter Prototyp des Laparoskopietrainers, mit Verbesserung des Zugangs via Neoprenhaut \ref{fig:phantomI}.\\
        \clearpage

        Durch den Kontakt zu Dr. Ester Stenau \cite{esther} konnten typische Bewegungsabläufe in minimalinvasiven Eingriffen abgeklärt werden, sowie Sinnvolle Benamungen der zu generierenden Daten. Dies ergab dass chirurgen meist mit 3 verschiedenen Arten von Bewegungen arbeiten:
        \begin{itemize}
            \item Zoom-Bewegungen\\
                Eintauchen des Endoskops in den Patienten bzw. das Herausziehen des Endoskops aus den Patienten.
            \item Rotation um die Patientenachse\\
                Also bei Cholezystektomien die bewegung um den liegenen Patienten herrum. \\
                (engl. Yaw)
            \item Rotationen um die Endoskope Achse \\
                Eine Drehung des Endoskopes selbst, innerhalb des Trokars.\\
                (engl. Roll)
        \end{itemize}

        Andere Bewegungen können durchausvorkommen sind aber im Testaufbau vorerst zu vernachlässigen, da die Hauptzweck darin bestehen soll für Algorithmen taugliche Daten zu generieren. Dies benötigt beispielsweise das langsame auftauchen / abziehen der Effekte. Die Videos sind Entsprechend so anzupassen dass besagte Verschmutzungen (siehe Grundlagen \ref{sec:Fundamentals}) auf einem Klaren Bild starten um dieses Dann bis zur unkenntlich zu stören.\\
        Als Ergebnis sollen Daten enstehen auf denen der Algorithmus so getestet werden kann das ein erstes Eingreifen (quasi untere Erkennungsschwelle) sowie ein Abbrechen des Algorithmus beobachtet werden kann (obere Erkennungsschwelle). \\
        Diese schwellen sollen die Möglichkeiten der Bearbeitung aufzeigen und so deutlich machen in welchem Maße die Verschmutzungen erkannt und auch (unabhängig davon) beseitigt werden können.

        \begin{figure}[H]
            \centering
            \includegraphics[width=0.8\textwidth]{pics/setups/Laparoscope_box.jpg}
            \caption{Gesamt Aufnahme des gebauten Laparoskopietrainers}
            \label{fig:phantomI}
        \end{figure}

        Einstiche im Neoprenstoff für die Einbringung der Werkzeuge in das künstliche Peritoneum \ref{fig:phantomII}.
        Die Einstiche sind aufgrund der Öffnung der Box relativ nah zusammen, in realen Operationen (Cholezystektomien) wird das Endoskope meist durch den Bauchnabel eingestochen. Dies hat hauptsächlich ästhetische gründe und dient dazu die größte Narbe der Operation zu verbergen, ermöglicht aber auch einen besseren Überblick über die weitern eingebrachten Werkzeuge. Diese werden näher an der Gallenblase in den Körper eingebracht, wie Abblidung \ref{fig:stabbing} darstellt.\\

        \begin{figure}[H]
            \centering
            \includegraphics[width=0.8\textwidth]{pics/usable_unsure/einstichmuster.jpg}
            \caption{Plazierung der Werkzeuge und des endoskopes bei Cholezystektomien}
            \label{fig:stabbing}
        \end{figure}

        Zur Verwendung kommen nur 2 Werkzeuge, ein Kauter und eine klemme. Um die benötigten Daten aufzuzeichnen  wird theoretisch nur der Kauter benötigt, aber aus rein praktischen gründen wird die Klemme zum halten und bewegen der Organe verwendet. \\
        Die auf Bild \ref{fig:phantomII} Einstiche zu sehenden Einstiche kommen daher von den Werkzeugen (unschwer am Blut auf dem Neopren zu erkennen) und links im Bild der Einstich der verschiedenen Laparoskope.

        \begin{figure}[H]
            \centering
            \includegraphics[width=\textwidth]{pics/setups/Laparoscope_box_cover.jpg}
            \caption{Neoprensstoff im Deckel des Trainers, mit Einstichen der Werkzeuge und des Endoskopes}
            \label{fig:phantomII}
        \end{figure}


        Beim Zweiten Prototyp werden frische Organe verwendet die am Morgen des Versuchstages geschlachtet wurden. Die Frische der Organe ist wichtig da so die Feuchtigkeit des Gewebes erhalten bleibt.

        \begin{figure}[H]
            \centering
            \includegraphics[width=\textwidth]{pics/setups/organs_marked.png}
            \caption{Organe vor der Videodatengenerierung}
            \label{fig:organs}
        \end{figure}

        \begin{table}[H]
            \centering

            \begin{tabular}{l|c|c}	%all left aligned
                \textbf{RefNr} & \textbf{Organ} & \textbf{Bemerkung} \\
                \hline
                \textbf {1} & Leber & Hämerperfusioniert\\
                \textbf {2} & Magen & inklusive Drüsen\\
                \textbf {3} & Milz & Hämerperfusioniert\\
                \textbf {4} & Lunge & beide Lungen\\
                \textbf {5} & Speise- \& Luftröhre & komplettes \\
                \textbf {6} & Zunge & gesamter Kehlkopf\\
                \textbf {7} & Fettnetz & Omentum Majus \\
            \end{tabular}
            \caption{verwendete Organe des 2ten Testaufbaus\label{tab:organlist}}
        \end{table}

        Da die Organe (Tabelle \ref{tab:organlist} und Bild \ref{fig:organs}) von mehreren Tieren stammen wird darauf geachtet das nur jeweils zusammenpassende bzw. zusammenhängende Organe gleichzeitig im Trainer verwendet werden.\\
        Die meist kauterisierten Organe des Versuches sind Leber in Kombination mit der Gallenblase, da jedoch nur eine der Lebern noch eine Gallenblase enthielt wurde im laufe der ''Operationen'' auf unter anderem die Lunge ausgewichen, da diese die benötigte Feuchtigkeit aufwies die für sinnvollen Rauch und Dampf notwendig ist.

        \begin{figure}[H]
            \begin{subfigure}[b]{\textwidth}
                \centering
                \includegraphics[scale=0.4]{pics/setups/setup2_schematic.png}
                \caption{Schematische Darstellung des Versuchsaufbaus\\
                    (1)Endoskope, (2)Neopren, (3)Optik, (4)Elektrokauter
                }
                \label{fig:MK_II_schematic}
            \end{subfigure}
            \hspace{1cm}
            \begin{subfigure}[b]{\textwidth}
                \centering
                \includegraphics[scale=0.4]{pics/setups/setup2_real.png}
                \caption{Bild aus der Generierung der Testdaten mit einem Stereoendoskope,\\
                    (1)Endoskope, (2)Neopren, (3)Optik, (4)Elektrokauter
                }
                \label{fig:MK_II_real}
            \end{subfigure}
            \caption{Aufbau der Datengenerierung}
            \label{fig:MK_II}
        \end{figure}

        In Tabelle \ref{tab:usedHW_II} finden sich die Referenzen zu den Bildern \ref{fig:MK_II_real} und \ref{fig:MK_II_schematic}.\\

        \begin{table}[H]
            \centering
            \begin{tabular}{c|l|l}
                \textbf{REF} & \textbf{Gerät} & \textbf{Beschreibung} \\
                \hline
                \textbf {1} & Endoskope & Aufnahme Element\\
                \textbf {2} & Neopren & Haut für Einstiche\\
                \textbf {3} & Optik & einzuführendes Objektiv des Endoskopes\\
                \textbf {4} & Kauter & Verbrennung Werkzeug\\
            \end{tabular}

            \caption{benutzte Geräte und Werkzeuge\\ Nummern referenzieren die Skizze \ref{fig:MK_II_schematic} und Abbildung \ref{fig:MK_II_real}\label{tab:usedHW_II}}
        \end{table}

        Der Dargestellt Aufbau beschreibt den größten Teil er Daten Generierung insbesondere im Bereich der realistischen Daten.\\
        Zusätzlich zu diesen, mit ''Dynamisch'' in den Datensätzen sowie Tabelle \ref{tab:testData} gekennzeichnet, wurden auch noch ''statische Daten'' (mit ''static'' gekennzeichnet \ref{tab:testData}) erzeugt. \\
        Diese statischen Daten konnten unter Zuhilfenahme eines handelsüblichen Wasserkochers generiert werden.\\
        Der Dateiname gibt entsprechend an wie die Daten generiert wurden, bzw was darauf zu sehen ist, sowie die Art und den Aufnahmekanal des verwendenden Endoskopes. \\
        \newline
        Die statischen Daten haben den Zweck die bereits angesprochenen Erkennungsschwellen zu identifizieren. Mit deren Hilfe sollte es insbesondere möglich sein für jede Erkennung jeder Verschmutzung grenzen festzulegen.\\
        Desweiteren können so Einflüsse von verschiedenen Bewegungen und und deren Auswirkungen auf die Daten evaluiert werden.
        \clearpage

        \begin{figure}[h]
            \vspace{5cm}            % sieht halt trotzdem wie die letzte scheiße aus
        %%%%%%%%%%%%%%%%%%%%%%%%%%%%%%%%% FIRST ROW
            \begin{subfigure}[b]{0.32\textwidth}     % eindritteln der Textfläche für plazierung neben einander
                \centering
                \includegraphics[width=\textwidth]{pics/ownData/old_dynamicSmoke_s0.jpg}
                \caption{Gallenblasen mit Zange 1 }
                \label{fig:dynamic_1}
            \end{subfigure}
            ~
            \begin{subfigure}[b]{0.32\textwidth}
                \centering
                \includegraphics[width=\textwidth]{pics/ownData/old_rotation_s0.jpg}
                \caption{Gallenblasen Rand 1}
                \label{fig:dynamic_2}
            \end{subfigure}
            ~
            \begin{subfigure}[b]{0.32\textwidth}
                \centering
                \includegraphics[width=\textwidth]{pics/ownData/old_stain_s0.jpg}
                \caption{Lunge mit Dampf 1}
                \label{fig:dynamic_3}
            \end{subfigure}
        %%%%%%%%%%%%%%%%%%%%%%%%%%%%%%%%% second ROW
            \begin{subfigure}[b]{0.32\textwidth}     % eindritteln der Textfläche für neben einander
                \centering
                \includegraphics[width=\textwidth]{pics/ownData/old_dynamicSmoke_s1.jpg}
                \caption{Gallenblasen mit Zange 2 }
                \label{fig:dynamic_1}
            \end{subfigure}
            ~       % when I am Right, this should mean that they are placed next to each other
            \begin{subfigure}[b]{0.32\textwidth}
                \centering
                \includegraphics[width=\textwidth]{pics/ownData/old_rotation_s1.jpg}
                \caption{Gallenblasen Rand 2}
                \label{fig:dynamic_2}
            \end{subfigure}
            ~
            \begin{subfigure}[b]{0.32\textwidth}
                \centering
                \includegraphics[width=\textwidth]{pics/ownData/old_stain_s1.jpg}
                \caption{Lunge mit Dampf 2}
                \label{fig:dynamic_3}
            \end{subfigure}
        %%%%%%%%%%%%%%%%%%%%%%%%%%%%%%%%% third ROW
            \begin{subfigure}[b]{0.32\textwidth}     % eindritteln der Textfläche für neben einander
                \centering
                \includegraphics[width=\textwidth]{pics/ownData/old_dynamicSmoke_s2.jpg}
                \caption{Gallenblasen mit Zange 3 }
                \label{fig:dynamic_1}
        \end{subfigure}
        ~       % when I am Right, this should mean that they are placed next to each other
        \begin{subfigure}[b]{0.32\textwidth}
            \centering
            \includegraphics[width=\textwidth]{pics/ownData/old_rotation_s2.jpg}
            \caption{Gallenblasen Rand 3}
            \label{fig:dynamic_2}
        \end{subfigure}
        ~
        \begin{subfigure}[b]{0.32\textwidth}
            \centering
            \includegraphics[width=\textwidth]{pics/ownData/old_stain_s2.jpg}
            \caption{Lunge mit Dampf 3}
            \label{fig:dynamic_3}
        \end{subfigure}

        \caption{Bilder der Dynamischen Daten}
        \label{fig:DataAcq}
    \end{figure}

    \clearpage

    Die Verbesserungen in \ref{tab:improvements} erzeugen Daten, die für eine Weiterverarbeitung in Frage kommen, jedoch noch qualitativ echten Daten gegenübergestellt werden müssen. \\

\subsubsection{Reale Daten}
    Durch einen Kontakt bei der Firma NewTec GmbH wurde es möglich reale intraoperative Bilder von Laparoskopien zu erhalten.
    Diese Bilder hatten maßgeblichen Einfluss auf die Entstehung der Aufnahmen des zweiten Prototypes und dienten als Referenz in Bezug auf Realismus.\\
    \newline
    Die Bilder selbst unterliegen jedoch dem Datenschutz der Patienten und dürfen deshalb nicht in der Arbeit veröffentlicht werden.\\
    \newline
    Der Ausschnitt \ref{fig:wolfVid} einem auf Youtube veröffentlichten Werbe-Video der Firma RICHARD WOLF GmbH von 2014 \cite{WolfVid} zeigt ein Bild das ebenfalls mit dem Stereoendoskop 3DHD \cite{3DHD} aufgenommen wurde und den Aufnahmen Uwe Widmaiers \cite{uwe} stark ähnelt.\\
    Das Bild zeigt einen Elektrokauter, der zum schneiden von Gewebe verwendet wird und Ansätze von Surgical smoke, welcher sich im Pneumoperitoneum ausbreitet.

    \begin{figure}[h]
        \centering
        \includegraphics[width=\textwidth]{pics/usable_unsure/wolfVid.jpg}
        \caption{
            Aufnahme eines realen Laparoskopische Eingriffes \\
            Aufgenommen mit einem 3DHD \cite{3DHD}, \\
            Werkzeug im Bild ist ein monopolarer Elektrokauter.\\
        }
        \label{fig:wolfVid}
    \end{figure}


\newpage
\section{Datenbearbeitung}
    Die Bearbeitung der generierten Daten teilt sich in mehrere Teile.
    Zuerst müssen die Daten aufgrund ihrer Größe verwertbarer gemacht werden - mit Hilfe von Konversion des Dateiformates sowie einer Reduktion der Auflösung.\\
    Im Anschluss werden die Bilder unter Zuhilfenahme von Histogrammen verglichen und bearbeitet. \\
    Eine Implementation von Rauch- oder Dampferkennung ist nicht mehr Teil der Arbeit.

    \subsection{Tools und Sprachen}
        Die Verarbeitung der Daten erfolgt auf einem Dell Latitude \cite{laptop} als Host-System, in Verbindung mit einer Ubuntu18.04 virtuellen Maschine die mittels VMware-Player auf dem Hostsystem die Verwendung der Bash und anderen Programmierwerkzeugen ermöglicht. \\
        Die gewählten Sprachen sind: \\
        \begin{itemize}
            \item Python3.6.7,
            \item Bash 4.4.19  und
            \item Octave 4.2.2 .
        \end{itemize}

        Insbesondere für Python wurden weitere Bibliotheken verwendet, welche alle via Python-pip:\\
        \lstinline[language=python, numbers=none ]{python3 -m pip install <package> [<package>] } \\
        installiert werden können. Die Packete sind: \\

        \begin{itemize}
            \item numpy         1.16.2
            \item openCV-python 4.1.0.25
            \item matplotlib    3.3.0
            \item pycparser     2.17
            \item scipy         1.2.1
        \end{itemize}
        % additionals : argparser and csv (parser)

        Die openCV-python Bibliothek wurde verwendet, da es sich hierbei um eine C++-Bibliothek mit einem Python-wrapper handelt, die Bildbearbeitung in Bildern sowie Videos zulässt. In Verbindung mit Numpy und matplotlib ist es so möglich schnell zu prototypischen Ergebnissen zu gelangen.\\
        Python als geskriptete Sprache erlaubt eine schnelle Implementation des Codes, jedoch bleiben die Ausführungszeiten hinter derer von kompilierten Sprachen wie C++ zurück. Da die openCV-Bibliothek für python nur ein wrapper um die eigentliche C++ Implementierung ist, kann so erwartet werden, dass sich etwaiger Portierungsaufwand gering hält. Desweiteren ist openCV eine in sowohl in Wissenschaft also auch in der Industrie weit verbreitete Bibliothek für Bildbearbeitungen.\\
        Matplotlib und numpy sind mathematische Bibliotheken, die bei der Erstellung von beispielsweise Histogrammen vorgefertige Algorithmen mitbringen, auf die zugegriffen werden kann. \\
        Sowohl numpy, welches eigene Datentypen mitbringt, als auch Matplotlib, das Bilder aus openCV in Plots darstellen kann, sind kompatibel mit OpenCV und machen eine kombinierte Verwendung einfach.\\
        \newline
        Dennoch wurde im Laufe der Arbeit auch auf Octave zurückgegriffen, einer unter GNU GPLv3 lizenzierten Software für numerische Berechnungen, um schneller an Ergebnisse zu kommen als über Python mittels Opencv und numpy.\\
        Ein weiterer Vorteil von Octave ist der produzierte Code, der sehr mathematisch zu lesen ist und so eine einfache Übersetzung von Formeln in Quelltext ermöglicht.\\

    \subsection{Vorverarbeitung}

        Die generierten Daten dienen im Folgenden als Basis für die Bildanpassungsalgorithmen.\\
        Da das Ergebnis der Basisdatengenerierung aus Videodateien in .avi-formaten besteht ist es vorerst nötig diese zu konvertieren, anzupassen und Bilder zu generieren.\\
        Um die Größe der Videos auf eine einfach zu verwendbare Größe zu konvertieren wurden die .avi Dateien mittels FFMPEG zu .mp3 Dateien mit einer Auflösung von 800x600px konvertiert.\\
        \newline
        Ziel der Vorverarbeitung ist es die aus den Versuchen generierten Videorohdaten für die Implementation vorzubereiten.
        Beispielsweise werden für die ersten Versuche der Bildbearbeitung mit den bereits erwähnten Gamma-, Kontrast-, Helligkeitsfiltern einzelne Bilder (Frames) der Videos benötigt.\\
        Die Vorverarbeitung soll die Daten aufbereiten um komfortables Arbeiten mit ihnen zu ermöglichen, indem die Videos in Standardformate mit kleineren Größen konvertiert werden. Mit den Bildern wird ebenso verfahren.\\
        \newline

        Durch die Verwendung einer Bash auf einem Linuxsystem kann mittels ffmpeg in wenigen Zeilen die Konversion aller Videodateien angestoßen werden \ref{code:vidConf}.

\begin{lstlisting}[language=bash, numbers=none, caption={converting avi to scaled mp4}, label={code:vidConf}]
    for i in ./VideoFiles; do
        ffmpeg -vf scale=800x600 -i $i $i.mp4
    done
\end{lstlisting}

        Das Ergebnis der Umwandlung sind deutlich kleinere Videos, in einem Format, dass die Ziel-Sprachen Python3.7, Octave, C++ einfach und schnell verwenden können\ref{tab:formatVid}.
        \begin{table}[h]
            \centering
            \begin{tabular}{c|c|c|c}
                newonversion & Format & Dateigröße & Geometrie\\
                \hline
                input: & AVI & 837MB & 1920x1080\\
                output: & MP4 & 9,4MB & 800x600 \\
            \end{tabular}
            \caption{Informationen der Konversion\\
                File:\\
                Stereo-LiverTest-Smoke-RotLapAxis-\#1-Right.avi
                \label{tab:formatVid}
            }
        \end{table}

        der Coder \ref{code:vidConf} wandelt die Eingabevideos in ein besser handhabbares Format um \ref{tab:formateVid}. \\
        Aus den MP4 Videos ist es nun einfach Bilder zu extrahieren, die Herangehensweise hierbei ist jede Sekunde des Videos einen Screenshot zu erstellen.\\
        \newline
        Das Skript \textbf{shotGrabber.py} \ref{code:shotGrabber} dient als Folgeskript um aus den generierten Videodaten letztendlich Bilder zu erstellen.\\
        \lstinputlisting[language=Python, caption={script zur sekündlichen Erstellung von Screenshots}, label={code:shotGrabber}, firstline=14, lastline=28]
        {../../work/scripts/shotGrabber.py}

        Die Vorgehensweise dabei ist es alle Videos anzusehen und nach jeweils einer Sekunde einen Screenshot abzuspeichern. Der Screenshot ist hierbei der aktuelle Frame des Videos \ref{code:shotGrabber} Zeile 12.\\
        Gespeichert werden die Bilder jeweils in einen Ordner mit dem Namen des Videos, die Files selbst haben einen einheitlichen Namen: \textit{''shot\_\%d.jpg''}, wobei das \textit{\%d} die Sekunde der Aufnahme codiert. \\
        Das Ergbnis  der Bilddatengenerierung sind 1244 (221 Originale aus \cite{uwe} und 1023 Bilder aus den selbst generierten Daten). \\
        \newline
        Diese Bilder dienen im Anschluss als Eingabe für ebenso viele Histogramme.

    \subsection{Histogramme als Vergleichsgrundlage}
        Mit Hilfe des \ref{code:histPY_calc} Scriptes, können Histogramme erzeugt werden, die Auskunft über die farbliche Zusammensetzung der Bilder geben.

        \lstinputlisting[language=Python, caption={Histogram Generator Skript, Methode:calcHist}, label={code:histPY_calc}, firstline=22, lastline=44]
        {../../work/scripts/hist.py}

        Die Methode zeigt das iterative kumulieren der Werte jedes Pixels, die Arrays \lstinline{hhist_b, hist_g, hist_r} enthalten die Anzahl der \lstinline{uint8_t} Werte für die Farbkanäle Rot, Grün und Blau des RGB Formates.\\
        Weitere Methoden sorgen dann für die Generation von csv-dateien für die Weiterverarbeitung \ref{code:histPY_csv} oder für die Darstellung als Kurve \ref{code:histPY_fun}. \\
        \lstinputlisting[language=Python, caption={Histogram Generator Skript, Methode:createCSV}, label={code:histPY_csv}, firstline=80, lastline=90]
        {../../work/scripts/hist.py}
        \lstinputlisting[language=Python, caption={Histogram Generator Skript, Methode:histLong}, label={code:histPY_fun}, firstline=93, lastline=129]
        {../../work/scripts/hist.py}

        Der Ausgabeplot für das Kommando:\lstinline{gram shot_24.jpg -l s24.csv -o hist_24.png} sieht wie folgt aus \ref{fig:histInUse}.\\

        \begin{figure*}[h]
                \includegraphics[width=\textwidth]{pics/code/histInUse.png}
                \label{fig:histInUse_code}
                \caption{Hilfe und Verwendung des Histogram-Erstellungs Scripts}
        \end{figure*}

        \begin{figure*}[h]
            \centering
            \begin{subfigure}[b]{0.4\textwidth}
                \includegraphics[width=\textwidth]{pics/code/shot_24.jpg}
                \caption{Eingabe Datei von \ref{fig:histInUse}}
                \label{fig:histInUse_in}
            \end{subfigure}
            ~
            \begin{subfigure}[b]{0.4\textwidth}
                \includegraphics[width=\textwidth]{pics/code/hist_24.png}
                \caption{Ausgabe Datei von \ref{fig:histInUse}}
                \label{fig:histInUse_out}
            \end{subfigure}
            \caption{Ein und Ausgabe sowie Screenshot des Prozesses des Histogramgenerators \ref{fig:histPYcode}}
            \label{fig:histInUse}
        \end{figure*}

        Der Parameter \lstinline{ --createCSV} sorgt dafür dass die Arrays \lstinline{hhist_b, hist_g, hist_r} als csv-datei abgespeichert werden. So können später alle Farbwerte aufsummiert werden um ein Gesamthistogram über, beispielsweise, ein gesamtes Video zu generieren.\\
        Die resultierenden CSV Dateien werden dann mittels des Scriptes \ref{code:collectorPY} zusammengefasst.\\
        Gruppiert wird nach Quelle: ein Histogram beschreibt die Zusammensetzung aller realen Daten die durch Uwe Widmaier \cite{uwe} aufgezeichnet wurden und ein weiteres über die Daten die im Laborversuch selbst erstellt wurden.\\
        Auf der Grundlage dieser Histogramme können die erstellten Daten mit den realen Daten verglichen und evaluiert werden.\\
        Der explizite Code zu diesem Vorgehen findet sich im \ref{sec:Appendix}, sowie eine Liste der generierten Daten. \\
        Die Ergebnisse der hier beschriebenen Algorithmen finden sich im Kapitel Ergebnis \ref{sec:Results}.\\

        \newpage
         \subsection{Anwendung von Bildanpassungsalgorithmen}

            Die unter \ref{sec:Fundamentals} beschriebenen Algorithmen wirken sehen im code folgendermassen aus \ref{code:plotIt}.
            Dieses Skript produziert die Ausgabebilder in \ref{fig:imageAdjust}. \\

            \begin{figure}[H]
                \begin{subfigure}[h]{0.4\textwidth}
                    \includegraphics[width=\textwidth]{pics/eval/example_pic_0.jpg}
                    \label{fig:contrast}
                    \caption{
                        Originales Eingabebild
                    }
                \end{subfigure}
                ~
                \begin{subfigure}[h]{0.4\textwidth}
                    \includegraphics[width=\textwidth]{pics/eval/brightness.jpg}
                    \label{fig:contrast}
                    \caption{
                        Änderung der Helligkeit um 10\%
                    }
                \end{subfigure}
                ~
                \begin{subfigure}[h]{0.4\textwidth}
                    \includegraphics[width=\textwidth]{pics/eval/gamma.jpg}
                    \label{fig:contrast}
                    \caption{
                        Änderung des Gamma-Wertes um 10\%
                    }
                \end{subfigure}
                ~
                \begin{subfigure}[h]{0.4\textwidth}
                    \includegraphics[width=\textwidth]{pics/eval/sigmoid.jpg}
                    \label{fig:contrast}
                    \caption{
                        Änderung des Kontrastes mittels Sigmoid-Funktion um 10\%
                    }
                \end{subfigure}
                \caption{Bildbearbeitung von intraoperativen Bildern\label{fig:imageAdjust}}
            \end{figure}

            Die Anpassung wird hier bereits über Histogrammatching vorgenommen, welches die kumulierte Summe der Bilder an eine Eingabefunktion anpasst.
            Dadurch ensteht eine Matrix die auf die Pixel des Bildes angewandt werden kann und so jeden Pixel anpasst.


        \subsection{Histogram Equalization}
            Für die Implementierung des im Grundlagen-Kapitel \ref{sec:Fundamentals} beschriebenen Histogram Matchings ist es nötig Histogramme der Eingabebilder zu erstellen. Die Implementierung dieser mathematischen Funktionen erfolgt aus Performance-Gründen mittels Octave.\\
            So kann die Histogrammgenerierung, die bisher mittels des python Skripts hist.py \ref{code:histPY_calc} erreicht wurde, verkürzt werden.\\
            \newline
            Der Unterschied, der den hohen Geschwindigkeitszuwachs ermöglicht, ist hierbei nicht der Wechsel der Sprache, sondern der Methodik.\\
            Während das python-Script iterativ alle Pixel aufsummiert, ist die Vorgehensweise des Octave Scripts vektoriell. Dies kann mit Hilfe der numpy-Bibliothek auch in Python realisiert werden\cite{numpy}.
            Der Zeitgewinn durch die Vektorielle Rechnung mittels Octave ist immens, er verringert sich von ca 4 Minuten (für RGB Histogramme ungefähr 12 Minuten)) auf unter 1 Sekunde auf dem verwendeten System. \\
            Den entsprechenden Code zeigt der Auszug \ref{code:histogram} aus histogram.m \ref{code:histogramM}.\\

\lstinputlisting[language=matlab, caption={octave implementierung der Histogramm Erstellung}, label={code:histogram}]
        {../../work/mathe/own/histogram.m}

        Diese Version des Codes erzeugt allerdings ein Histogramm über die Graustufen des Bildes um ein eindimensionales Array zu erhalten, während \ref{code:histPY_calc} für jeweils Rot, Grün und Blau eigene Histogramme erstellt.
        Für die Anpassung mittels Histogram-Matching wird im code \ref{code:histogram} ein eindimensionaler Vektor erstellt ( Zeile 5 ). Durch die Bildung des Mittelwertes wird so ein Monochromes Bild erstellt. Monochrome Bilder bestehen nur aus Graustufen und deren Histogramme sind entsprechenden eindimensionale Matrizen, sprich lediglich Vektoren.\\
