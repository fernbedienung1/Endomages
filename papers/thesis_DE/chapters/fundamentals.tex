\section{Verunreinigungen in Laparoskopischen Eingriffen}
    Laparoskopien sind minimal invasiven Eingriffe, bei denen ein Endoskope in ein Pneumoperitoneum eingeführt wird, um so Sichtkontakt zum Operationsgebiet zu erhalten. Die Bilder die das Endoskope hierbei aufnimmt und mittels Monitor den Chirurgen zur Verfügung stellt sind oft die einzige Möglichkeit zur Navigation innerhalb des Patienten.\\
    \newline
    Durch die Arbeit der Chirurgen, insbesondere durch den Einsatz von Kautern entstehen innerhalb des Pneumoperitoneums Verunreinigungen. Kauter erhitzen das Gewebe um die enden von beispielsweise Venen zu veröden, damit keine Blutung, wie etwa bei einem Schnitt, entsteht. Da es sich hierbei jedoch um eine Verbrennung von Gewebe handelt enstehe für Verbrennungen typische Nebeneffekte. \\
    \newline

    \subsection{Surgical Smoke}
        Der in der Literatur meist vertretene Effekt ist die Entstehung von Verbrennungsrauch (engl. \textit{Surgical Smoke}). Dieser surgical smoke blockiert die Sicht, da er sich im Pneumoperitoneum nicht verflüchtigen bzw. Abziehen kann. \\
        Die gestalten dieses Rauches reichen von dichten Schwaden bis zu einer Nebelartigen Diffusion die sich wie ein Film über das Kamerabild legt (siehe Rechts oben in Bild \ref{fig:realNoise}/(1)). \\
        Das entfernen dieses Rauches kann physisch mittels Absaugung oder Magnetischer Verdrängung erfolgen. Diese Methoden erfordern allerdings zusätzliche Anschaffungen von Geräten und teure Aufbauzeiten für die Chirurgen und deren Assistenten. 

    \subsection{Dampf und Beschlag}
        Das Kauterisieren von Gewebe hat weiterhin den Effekt das körperinterne Flüssigkeiten, wie Blut, Lymphflüssigkeit oder Fette verdampfen. Auch diese können das Pneumoperitoneum nicht verlassen und kondensieren an den kältesten punkten innerhalb des Patienten, den Instrumenten und dem Objektiv der Kamera. Zu sehen in der rechten unteren ecke von Bild \ref{fig:realNoise}/(2). \\
        
    \subsection{Verschmutzungen}
        Desweiteren kann es dazu kommen das sich Spritzer der oben genannten Flüssigkeiten auf dem objektiv der Kamera sammeln. Diese spritzer bilden tropfen auf dem Kamera Bild durch die nur teilweise bis gar nicht hindurch gesehen werden kann. Dem Chirurgen bleibt bei solchen verschmutzungen meist nichts anderes als das entfernen des Endoskopes um es zu reinigen. 
        
    \begin{figure}[h]
        \centering
        \includegraphics[width=0.5\textwidth]{pics/schmauch/noisesMarked.png}
        \caption{Verunreinigungen in Endoskope aufnahmen, \\
            \textbf{(1)}surgical smoke, \textbf{(2)}Beschlagen des Endoskopes durch aufsteigenden Dampf\\
            \textsc{[Eigene Aufnahme]}
        }
        \label{fig:realNoise}
    \end{figure}

%%%%%%%%%%%%%%%%%%%%%%%%%%%%%%%%%%%%%%%%%%%%%%%%%%%%%%%%%%%%%%%%%%%%%%%%%%%%%%%
%% PREQUISITES
%%%%%%%%%%%%%%%%%%%%%%%%%%%%%%%%%%%%%%%%%%%%%%%%%%%%%%%%%%%%%%%%%%%%%%%%%%%%%%%
\section{Grundlagen laparoskopischer Eingriffe}
    Um eigene Bilder von Laparoskopien zu erstellen müssen einige Aspekte beachtet werden, um zu gewährleisten dass die Generierten Daten echten Operationen ähnlich sind.
    Desweiteren sollten die Generierten Daten geeignet sein um mit ihnen Algorithmen zu entwickeln. \\
    Diese Eignung schließt bestimmte Aspekte ein, wie beispielsweise das langsame erscheinen von Verunreinigungen oder definierte Bewegungen der Kamera. \\
    \newline

    Die Befragung verschiedener Chirurgen ergab das insbesondere die Art der Kauterisation (sprich: Elektrisch, Thermisch oder etwa via Laser) maßgeblich ist für die Beschaffenheit und das aussehen des Rauches.\\
    Desweiteren bestimmt die Art des zu kauterisierenden Gewebes die Verschmutzungen die entstehen.\\
    \newline
    
    Das Generieren eines eigenen Datensatzes unterliegt deshalb einigen Restriktionen, insbesondere ist es nicht möglich aufnahmen von Menschen zu erzeugen.\\
    Dies führt dazu das ein Ersatz gefunden werden muss, potentielle Kandidaten hierfür sind:\\

	\paragraph{\textbf{Mäuse} \textit{(Mus musculus)}\\}
        Haben eine große Ähnlichkeit zum menschlichen Organismus, sie werden deshalb häufig für Versuche im Medizinischen Bereich eingesetzt. Jedoch Beschränkens sich diese versuche häufig auf die Verträglichkeit von Medikamenten oder anderer, weitgehend konventioneller chirurgischer eingriffe.
        Mäuse sind jedoch aus rein physischen Gründen für die Erstellung der Daten nicht geeignet, da es schlichtweg unnötig schwierig ist Endoskope in sie einzubringen.\\
        \cmt{zu ''leger''?}

        
	\paragraph{\textbf{Organe von Rindern} \textit{(Bos primigenius taurus)}\\}
        Organe von Rindern unterscheiden deutlicher von Menschlichen Organen, sind aber insbesondere wenn es um viszerale Organe geht deutlich verfügbarer und einfacher zu handhaben als Innereien von Mäusen. \\
        Die Verfügbarkeit sowie einfache Handhabung von Rinderinnerein  macht sie zu einer guten Wahl für erste Tests, insbesondere für die Verwendung in Laparoskopischen Trainern, wie sie von angehenden Chirurgen zum trainieren verwendet werden.\\
        Für weitergehende Tests sollten jedoch Organe verwendet werden die Menschenähnlicher oder Menschlich sind, um sicher zu gehen dass das Gewebe auch realistisch verbrennt und entsprechenden Rauch, Dampf und andere Verschmutzungen produziert.
	\newline	 \\

	\paragraph{\textbf{Organe von Schweinen}\textit{(Sus scrofa domesticus)}\\} 
        Die Brücke zwischen Handhabung, Verfügbarkeit und Ähnlichkeit zu Menschlichen Organen schlagen Schweine.\\
        Während es bei schweinischen Innereien seltener als bei Rinderinnereien zum Verzehr durch den Menschen kommt, ist es trotzdem möglich an Frische viszerale Organe von Schweinen zu kommen, da diese andern falls zu Hundefutter oder ähnlichem verarbeitet werden und damit häufig ein Abfall Produkt von Schlachtern sind.\\
        Wie auch bei Mäusen ist es üblich Medikamente oder andere verfahren aus Medizin und Medizintechnik an Schweinen zu testen bevor sie am Menschen freigegeben werden.
        Die Ähnlichkeit und Eignung von Schweinen zeigt sich unter anderem daran dass sie auch auch in Medizinischen Ausbildungen häufig als erste Instanz für Chirurgen dienen, die in Lebendigen Organismen operieren \cite{swineModel}. \\
        Bei Schweinen ist zudem der Aufbau des Kardiovaskulären Systems sowie die Anordnung der viszeralen und digestiven Organe analog zum Menschlichen Aufbau.\\
        Folglich sind Schweine die Optimalen Kandidaten für die Generierung der Datenbasis. \\
    \newline

	\paragraph{Organe\\}
        Endoskope werden häufig in Laparoskopien eingesetzt um so an den visceralen Organen operieren zu können.
        Zu diesen Organen gehören unter anderen:

		\begin{itemize}
			\item Gallenblasen( Vesica biliaris),
			\item Lebern( Hepar ),
			\item Mägen ( Gaster )
			\item oder dem Gastrointestinaltrakt( Tractus digestorius ).
		\end{itemize}

        Insbesondere Gallenblasen sind mit 200.555 Cholezystektomien die im Jahr 2017 \cite{destatis} durchgeführt wurden, typische Organe die auf endoskopischen Aufnahmen auftauchen. Cholezystektomien gehören damit zu den häufigsten Laparoskop ische Eingriffen und sind Standard in vielen Krankenhäusern.


        Die Gallenblasen ist mit der Leber verbunden ist selbst für Laien durch ihre Form und Farbe meist gut zu identifizieren.
        Die schematische Darstellung \ref{fig:schema_biliaris} zeigt die Position der Gallenblase beim Menschen. \\
        Die Leber hat an der stelle an der die Gallenblase aufsitzt eine Mulde, die sogenannte Gallenblasengrube \textit{(Fossa vesicae biliaris)}.

        \begin{figure}[h]
            \centering
            \includegraphics[width=0.5\textwidth]{pics/net/vesicaBiliaris.png}
            \caption{ Schematische Darstellung einer Gallenblasen (grün), die auf der Leber sitzt.}
            % sourch http://medbox.iiab.me:3000/wikipedia_en_medicine_2018-01/A/Cystohepatic_triangle.html
            \label{fig:schema_biliaris}
        \end{figure}

        Bild \ref{fig:real_biliaris} zeigt eine schweinische Gallenblasen, die minimal invasiv operiert wird. Der Kauter verbrennt Gewebe der Gallenblasengrube um die Gallenblase von der Leber zu lösen.

        \begin{figure}[h]
            \centering
            \includegraphics[width=0.5\textwidth]{pics/ownData/vesicaBiliaris_REAL.jpg}
            \caption{Reales Bild einer Gallenblase eines Schweines\\
                \textsc{[eigene Aufnahme]}
            }
            \label{fig:real_biliaris}
        \end{figure}

        Ein weiterer Punkt der Selektion der Organe ist die Qualität bzw. die Frische.
        Frische Organe sind notwendig da unter anderem Feuchtigkeit ein wichtiger Aspekt ist der vor allem die Bildung von Rauch und Dampf beeinflusst.
        Eine weitere Option um die Feuchtigkeit der Organe zu gewährleisten ist das externe einbringen von Flüssigkeiten.\\
        Auch Körperinnere Luftfeuchtigkeit und Temperatur wirkt sich auf das Beschlagen der Kamera aus sowie auf die Bewegungen und Dispersion des Rauches.\\
        \newline
        Ausserdem bringen O ...... Chemische Reaktionen die das gewebe oder dessen zusammensetzung verändern sollen so weit wie möglich vermieden werden.
        Verwesung, Verfaulung und andere Gärungsprozesse werden in der Medizin als Nekrosen bezeichnet 

	\subsection{Benötigte Eigenschaften}

        Eigenschaften der Video Daten die als Grundlage für die Implementation dienen sollen betreffenden nicht nur die bereits erwähnte Qualität und Realismus, sondern ausserdem ein Muster für das auftauchen der Verschmutzungen.\\
        Die benötigten Verschmutzungen sind:\\
        \newline
		
	\begin{itemize}
		\item Surgical Smoke,
		\item Dampf,
		\item Fleckenbildung von:
		\begin{itemize}
			\item Blut,
			\item Wasser,
			\item Lymphflüssigkeit.
		\end{itemize}
	\end{itemize}
	
    Als Vorbild für die Generation eigener Daten dienen Videos von Verschmutzungen des \gls{kkhgzkru}.\\
    Diese Videos zeigen verschiedene Arten von Rauch (verschiedene Kauterisationsmethoden) sowie Instrumente im Operationsbereich. 
    Ausserdem zeigen diese Video Aufnahmen reale Bewegungen des Endoskopes, des Patienten, sowie anderer Werkzeuge.\\
    \newline

    \cmt{Videos werden unten beschrieben? In Methodik?}
    \subsection{Kauterisierungsmethoden}
        Die Moderne Chirurgie kennt mehrere Kauterisationsmethoden für Gewebe
        \begin{itemize}
            \item Thermische Kauterisation 
            \item Diathermie 
            \item Laserschnittverfahren
        \end{itemize}

        \cmt{bescheiden wie die funktionieren und was für Schmutz raus kommt}


\section{Bild Bearbeitung}
    
    Aktuelle Literatur beschäftigt sich hauptsächlich mit der Verbesserung des Gesamtbildes, [\cite{vision}, \cite{darkChannel}, \cite{modelBased}, \cite{imageDehaze}].\\
    Wobei hier der Fokus insbesondere auf der Entfernung von surgical smoke liegt. Basierend auf dessen Eigenschaften hinsichtlich Kontrast und Farbzusammenstellung.

\subsection{Erkennung der Verschmutzungen}
    Um die Bilder des Endoskopes so wenig wie möglich zu verändern und somit zu verfälschen ist es Ziel der Arbeit Verschmutzungen zu erkennen. 
    Dies ermöglicht Anwendung von unterschiedlichen Bildbearbeitungsfiltern auf unterschiedliche Verunreinigungen im Bild, sowie das unverändert lassen von nicht verschmutzten Arealen im Bild. \\
    \newline
    Verschiedene Verschmutzungen erfordern jeweilige Erkennungsmechanismen. Die folgenden Verschmutzungen bzw. deren Erkennung ist in dieser Thesis berücksichtigt.

    \paragraph{Rauch\\}
        Eine weitverbreitete Anwendung für Raucherkennung findet sich abseits von Operationssälen in ''Riot-control'' Anwendungen für Überwachungskameras\cite{vision}.
        Die Aufnahmen der Überwachungskameras werden hierfür einige Frames lang analysiert um die Ausbreitung des Rauches als Grundlage für dessen Erkennung zu verwenden.

    \paragraph{Dampf \& Beschlagen\\}
        Laut Chirurgen des \gls{ADK} scheint es ein bisher wenig bearbeitetes Thema zu sein das beschlagen des Endoskopes abzumindern, auch Recherchen nach Lösungen für dieses Problem bestätigen diese Annahme. \\
        Durch die Bildung kleiner tropen verschwimmt das Bild und wird so unscharf. Verursacht wird dies durch Dampf, welcher beim erhitzen von Gewebe, o.ä. durch den Einsatz der Kauter ensteht. Das Erkennen von Unschärfe ist ein bereits wohlbekanntes Problem bei Kameras, wenn auch im Kontext der Scharfstellung von Bildern über Methoden des Automatischen Fokussierens.\\

    \paragraph{Flecken \& Spritzer\\}
        Flecken und Spritzer auf der Kamera sind ein nicht besonders häufiges, jedoch schwerwiegenden Problem. Da meist die einzige Möglichkeit um diese Verunreinigungen zu beseitigen darin besteht das Endoskope aus dem Patienten zu entfernen um es zu reinigen.\\
        Jedoch es sehr gut möglich diese Verschmutzungen zu erkennen, da sie in den bewegten Bildern des Endoskopes statisch auf dem Objektiv bleiben und meist eine Tropfenförmige, rund bis ovale Form besitzen.\\
        Abhängig von der Flüssigkeit die die Verschmutzung hervorruft kann es Sinnvoll sein innerhalb der Außenlinien des Tropfens Filter anzuwenden um etwaige verzerrende Effekte wie Brechungen abzumildern und zu entfernen.\\
        Ebenso können opake Verschmutzungen auftreten, bei denen das filtern oder bearbeiten nicht weiterhelfen kann.
        Für diese Fälle kann es fatal sein den Chirurgen falsche fakten anzuzeigen.  \\


\subsection{Evaluierung von Bilddaten}
    \begin{figure}[H]
        \centering
        \begin{subfigure}[h]{0.5\textwidth}
            \includegraphics[width=\textwidth]{pics/eval/example_pic_0.jpg}
            \label{fig:hist_example_pic}
            \caption{
                Beispielbild mit großen Dunkelstellen sowie hohem Rot Anteil.
            }
        \end{subfigure}
        ~
        \begin{subfigure}[h]{0.5\textwidth}
            \includegraphics[width=\textwidth]{pics/eval/example_histsSingle_0.jpg}
            \label{fig:hist_example_singles}
            \caption{
                Histogram des Bildes \ref{fig:hist_example_pic}
                Jeder Farbkanal ist einzeln dargestellt.
            }
        \end{subfigure}
        ~
        \begin{subfigure}[h]{0.5\textwidth}
            \includegraphics[width=\textwidth]{pics/eval/example_hist_0.png}
            \label{fig:hist_example_hist}
            \caption{
                Kombiniertes Histogram der des Bildes \ref{fig:hist_example_pic}\\
                \textbf{x-axis:}\hspace{1cm}Pixel wert\\
                \textbf{y-axis:}\hspace{1cm}Anzahl der Pixel
            }
        \end{subfigure}
        \caption{Bild eines Endoskopes und dessen Histogramme}
        \label{fig:hist_example}
    \end{figure}

\newpage
    
    Das Bild \ref{fig:hist_example} zeigt eine Laparoskopische Szene, und deren Histogram. 
    Histogramme beschreiben die Zusammensetzung von Farben in Bildern, indem sie die Pixel mit Farbwerten zählen und entsprechend die Anzahl (in Bild\ref{fig:hist_example_singles} als y-Achse), die dann über den wert des Pixels dargestellt werden (in Bild\ref{fig:hist_example_singles} als x-Achse).\\
    Dies ermöglicht den Vergleich der Bilder auf Basis ihrer Farbzusammenstellung.\\
    \newline
    Beispielsweise Dominieren in Bild \ref{fig:hist_example} die Schwarze linke Seite des Bildes sowie die Rötlichen Farben des Gewebes rechts im Bild.
    Die aus diesem Bild resultierenden Histogramme haben im dunklen Bereich des Bildes spitzen. Lediglich der Rot Kanal weißt eine höhere Anzahl von Pixeln im Bereich von größer 245 auf. Wobei hier jeder Farbkanal mit 8-bit codiert ist und damit der Maximalwert, also die hellste stärke der Farbe einem wert von $ 2^8 -1 = 255  $ entspricht.\\
    \newline
    Um im weiteren verlauf der Arbeit Bilder beschreiben zu können die selbst aus Datenschutz gründen nicht veröffentlicht werden dürfen, werden deren Histogramme als Maßstab für die Bewertung und den den Vergleich mit den eigens erstellten Daten herangezogen.

    \subsubsection{Rauch Erkennung}


