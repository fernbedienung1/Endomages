\section{testdaten Generierung}
    Im Rahmen der Arbeit wurden am 21.12.2018 Testdaten generiert \ref{tab:testData}, insgesamt 98GB Roh-Daten mit einer Dauer von 1 Stunde und 16 Minuten. \\
    Zusätzlich zu den 293MB realer Laparoskopischer Videos aus Günzburg (Gesamtlänge 5:44 min )

    Die generierten Daten dienen als Grundlage der für die Algorithmen, um \gls{overfitting} zu vermeiden.

    \subsection{Vergleich}
        Im Kapitel Methoden \ref{sec:Methods} beschrieben werden histogramme für den Vergleich der generierten Daten mit den Vorgabedaten verwendet, Bild \ref{fig:Cmp_hists}zeigt die Histogramme der eigens generierten Daten des Laborversuches vom 21.12.2018 und die der Realen aufnahmen von Dr.Uwe Widmand aus dem \gls{kkhgzkru}.\\
        Zu erwähnen ist das sich die generierten Histogramme auf unterschiedlich große Datensätze beziehen, während die aufnahmen aus Günzburg nur 293MB(Gesamtdauer: 5min 44s) groß sind,
        basiert das Histogram der Labor-Daten die verkleinert bereits 1GB(Gesamtdauer: 1h 15min 18s) groß sind.

        \begin{figure}[h]
            \centering
            \begin{subfigure}[b]{0.4\textwidth}
                \includegraphics[width=\textwidth]{pics/eval/cumulatedHISTOGRAM_OWNData.png}
                \caption{kumuliertes Histogram der eigenen Daten}
                \label{fig:cumsum_mine}
            \end{subfigure}
            ~
            \begin{subfigure}[b]{0.4\textwidth}
                \includegraphics[width=\textwidth]{pics/eval/cumulatedHISTOGRAM_REALData.png}
                \caption{kumuliertes Histogram der Daten aus Günzburg}
                \label{fig:cumsum_real}
            \end{subfigure}
            \caption{Vergleich der Generierten Daten mit realen Laparoskopien}
            \label{fig:Cmp_hists}
        \end{figure}

        Das Histogram der selbst generierten Daten (\ref{fig:cumsum_mine})gibt bereits Auskunft darüber wie die Bilder zusammengesetzt sind. Auffällig ist das im sich die meisten Pixel im Bereich von Intensitäten $ < 50 $ befinden, dies deutet darauf hin das die Bilder insgesamt dunkel ausfallen. Ein Indiz für große schwarze Anteile der Bilder kann eine Häufung von werten mit gleichem wert darstellen.\\
        (Pixel mit der Eigenschaft:$$ R \pm x = G\pm x = B\pm x$$ stellen Graustufen dar, wobei $ x $ stellvertretend für eine gewisse varianz steht.
        Da es sich bei den Histogrammen jedoch um kumulierte Daten handelt, sprich um ein Histogramm über mehrere Bilder könnten diese sich auch ausmitteln.\\
        Ein blick auf die Aufnahmen bestätigt jedoch die Annahme, denn die meisten der Bilder enthalten große schwarze Flächen, verursacht durch eine unzureichende Ausleuchtung des laparoskopischen Trainers. \\
        \newline 
        Zusätzlich zeigt das Histogram \ref{fig:cumsum_mine} einen erhöhten weiß-Anteil (Pixel mit $ (R=G=B) >= 240$ ). Dieser kann mit dem auftreten von Spiegelungen erklärt werden. Das eingespeiste Licht der Halogen Lichtquelle verursacht wenig diffuses Licht im Trainer und durch die kurzen Abstände zwischen Kamera und Werkzeug, bzw Organ enstehen auf den Feuchten Organen Reflexionen. Dies Reflexionen äussern sich durch weiße Pixel.\\

        Während hingegen die Pixel des Histograms der Realen Daten \ref{fig:cumsum_real} weiter in Richtung Mitte (Werte im Bereich $ 50 < x < 120 $) verteilt ist.\\
        Die Überlappung von Grün und Blau Anteilen in Kombination mit einer nach rechts verschobenen Rot kurve deutet darauf hin das es sich auch hier um dunkle Bilder handelt, Jedoch mit deutlichem Einschlag Rot Tones. \\
        Logischerweise lässt sich das durch dass durch das Aufgezeichnete Motiv erklären, den viszeralen Organen der Patienten.\\
        
        Vergleichend lässt sich hier folglich sagen das die Realen Daten \ref{fig:cumsum_real} Heller sein müssen als die selbst generierten.\\
        Dies lässt sich durch die Tatsache untermauern das bei der Generation der Daten lediglich eine von zwei möglichen Lichtquellen an die Kamera angeschlossen wurden. 
        In Ermangelung einer zweiten Lichtquelle war dies bereits bei der Generation der Daten abzusehen und entspricht daher den vorherigen Erwartungen.\\
        Die selbst generierten Daten sind im direkten Vergleich zu den realen Beispieldaten aus Günzburg deutlich dunkler, dies zeigen auch die Kumulierten Histogramme der beiden Datensätze \ref{fig:Cmp_hists}

        Für die ersten Versuche dieser reihe spielt die Helligkeit eine untergeordnete rolle, denn Rauch und andere Verunreinigungen können auch in den Dunkleren Bildern Deutlich mit bloßem Auge erkannt werden.\\
        Da die unzureichende Lichtquelle bereits bei der Erstellung der Daten als Problem erkannt wurde führte dies dazu dass die Aufnahmen der Organe mit geringen Abstand zum Objektiv aufgezeichnet wurden. \\

        \begin{figure}[h]
            \includegraphics[width=0.5\textwidth]{pics/ownData/darkpic.jpg}
            \caption{Bild mit schlechtem Licht und deshalb zu nah aufgenommener Leber}
            \label{fig:closeup}
        \end{figure}

        Auch dies steht nicht im direkten Konflikt des Ziels Schmutz aufzuzeichnen, jedoch entsteht so ein deutlich Sichtbarer unterschied zu den Realen Datensätzen.\\

        \subsubsection{quality}
            \begin{itemize}
                \item tools good
                \item movements different - nor bad or good
                \item static setup nice to work with in algorithm programming
                \item Eine einzuelne lichtquelle ist unzureichend!
            \end{itemize}

\section{Algorithms}
    \subsection{historgram matching}
        Das Bild \ref{fig:smokeHisto} zeigt die Anwendung der Histogram-matching Algorithmen auf ein Bild und deren resultierende Histogram kurven.\\
        \begin{figure*}[hp]
            \includegraphics[width=\textwidth]{pics/eval/smokeHist.pdf}
            \caption{Anpassungen der Bilder mittels Histogram-matching}
            \label{fig:smokeHisto}
        \end{figure*}

