
\section{Kritischer Blick auf die Arbeit}
\subsection{Ziele der Arbeit}
    \begin{table}[H]
        \centering
        \begin{tabular}{|l|l|}	%all left aligned
            \hline
            \textbf{Ziel} & \textbf{Status} \\
            \hline
            Generieren von Bildern die Störungen enthalten & Erreicht \\
            Erkennen von Störungen & Nicht Erreicht \\
            Entfernen von Störungen & Teilweise Erreicht \\
            Evaluation im Vergleich zu realen Bildern & Nicht Erreicht \\
            Evalutaion im Vergleich zu den Eingangsbildern & Nicht Erreicht \\
            \hline
        \end{tabular}
        \caption{Ziele der Arbeit\label{tab:goals}}
    \end{table}

    Tabelle \ref{tab:goals} Listet die ursprünglichen Ziele der Arbeit und deren Status. \\
    Aufgrund der aufwändigen Generierung der Testdaten war es im Umfang der Arbeit nicht möglich das eigentliche Ziel der Arbeit, eine live Verbesserung des Video Inputs zu erreichen.\\

\subsubsection{Erzeugung der Basisdaten}
    Der zeitliche Aufwand war bei der ursprünglichen Einschätzung des Zeitplans noch kein Teil der Arbeit, da fälschlicherweise davon ausgegangen wurde dass verrauchte Bilder öffentlich zugänglich sind bzw, leicht zu erlangen sind.\\
    Das Generieren der Daten hat insgesamt fast 3 Monate der 6 monatigen Arbeit eingenommen, die vorab nicht eingeplant waren. Jedoch sind die verunreinigten Eingabedaten für die Entwicklung und Erprobung der Algorithmen unabdingbar.\\
    Insbesondere die Organisation der Medizinischen Geräte gestaltete sich schwer, da diese zum einen Teil zur Hochschule Ulm bzw. dem DKFZ gehören (Endoskope) und zum anderen von der NewTec GmbH (Kauter \cite{sealer} und HF-Generator \cite{hfgen})im Rahmen von Projekten benötigt wurden.\\
    Bürokratische Hürden machten eine sinnvolle Kooperation unnötig schwer.\\
    \newline
    Ein weiteres unterschätztes Problem ist die Beschaffung der Organe. Organe in entsprechender Qualität, sprich: frische und mit Besonderheiten, wie beispielsweise einer noch nicht entfernten Gallenblase, sind schwer zu bekommen, da heutzutage nur noch weniger Metzger selbst schlachten und ihre Waren von nur wenigen Schlachthöfen bezogen werden.\\
    Um einen reibungslosen Ablauf zu garantieren müssen die Versuche entsprechend gut koordiniert werden, denn am Versuchstag können kleine Planungsfehler verheerend auf das Ergebnis Einfluss nehmen.\\
    \newline

\subsection{Qualität der Videodaten}
    Die Qualität der Videodaten wurde bereits im Laufe der Methodik mittels Algorithmen bestimmt, für einen Verlgeich wurden Histogramme zu Hilfe genommen.

\subsection{Was ist der Nutzen?}
    Der Gesamtnutzen der Arbeit besteht in vollständigen Datensätzen, die die weitergehende Entwicklung der Algorithmen ermöglichen. Diese Videodaten waren vorher nur in kleinen Mengen verfügbar, im Rahmen dieser Arbeit entstanden 75 Minuten Videomaterial. Besonderer Augenmerk der Daten liegt in der Eignung für die Entwicklung von Algorithmen. Dies zeigt sich daran das ein Großteil der Videos ohne Verschmutzungen startet, diese sich dann soweit ausbreiten, das mit bloßem Auge nicht mehr damit gearbeitet werden kann. \\
    Diese Eigenschaft kann in der Erprobung der Algorithmen verwendet werden um festzustellen ab welchem Grad es dem Algorithmus nicht mehr wirksam die Verschmutzungen erkennen bzw entfernen kann.

\section{Vergleiche}

\subsection{Vergleich zu Physikalischen Dispense Systemen}
    Die Entferung von Rauch im Pneumoperitoneum kann,
    Wie bereits in der Einführung angedeutet existieren bereits einige Methoden die sich mit der Thematik des Surgical smokes beschäftigen. Die meisten dieser Systeme stützen sich jedoch auf Hardware um zu verhindern, dass das Endoskop überhaupt erst einmal beschlägt, bzw. sich der Rauch im Pneumoperitoneum ausbreitet.\\
    Hierfür gibt es Sauger, die um die Werkzeuge angebracht werden \ref{fig:succer} \cite{succer} oder sogenannte Electrical Dispense Systems, die die Rauchpartikel electrostatisch aufladen, so dass diese zu den Wänden des Pneumoperitoneums magnetisch angezogen werden.\\
    Diese Electrical Dispense Systems haben im Vergleich zu rauch Absaugungen den Vorteil dass sie nicht gegen das Pneumoperitoneum arbeiten und nicht das eingepumpte Gas wieder entziehen. Deren jedoch nur mäßiger Erfolg und die dafür benötigte zusätzliche Hardware führen auch hier dazu das sich dieses System bis jetzt noch nicht durchgesetzt hat.

    \begin{figure}[H]
        \centering
        \includegraphics[width=\textwidth]{pics/usable_unsure/succer.jpg}
        \caption{Absaugung von Surgical smoke,\\
            mit dem Ziel den Rauch bereits beim Enstehen zu entfernen.
            \cite{succer}
        }
        \label{fig:succer}
    \end{figure}

    Diese Systeme finden sich jedoch selten in der Realität, denn sie sind mit extra Aufwand bei der Operation verbunden.\\
    Sauger, etc, bringen weitere Hardware in den Operationssaal, sowie unhandliche Kabel und Schläuche mit sich, was bereits als Problem diagnostiziert wurde, wärend der Hospitation im \gls{ADK}  Gespräch mit Dr Lotspeich \cite{Lotspeich}.\\
    \newline

\subsection{Softwarelösungen}
    Das einzige auffindbare System zur softwareseitigen beseitigung von Rauch in Operationsbildern stellt das \textit{Ultravision system von Alesi} dar. \cite{alesi}
    Jedoch finden sich nur wenige seriöse Quellen zu diesem system, insbesondere Vorher / Nachher Vergleiche des Systems wirken unprofessionell, da dass angebliche Eingabe video nicht dem Ausgabevideo entspricht \ref{fig:alesi}. \\

    \begin{figure}[H]
        \centering
        \includegraphics[width=\textwidth]{pics/usable_unsure/alesi.jpg}
        \caption{Beispiel der \textit{Ultravision} software. \cite{alesi} }
        \label{fig:alesi}
    \end{figure}

    Auch weiterhin lassen sich im Umfeld des \textit{Ultravision} Systems nur wenige Quellen oder verwertbare Daten finden, weshalb dieses System nicht weiter behandelt wird.

\section{Bedenken}

\begin{itemize}
	\item Quality of Tests / Setups
		\begin{itemize}
			\item TestBox
			\item Capturing Methoden
			\item Tools
		\end{itemize}
	\item with respect to the Cauterization kinds - thermal / hf
	\item Dr. Kunze - changes in smoke due to the carbonisation degree of the material
	\item Dr. Kunzes videos as extra eval data - to avoid overfitting
\end{itemize}

    \paragraph{Rauch- Arten}
        Die ersten Gedanken fast aller befragten Ärzte ergaben das sich Rauch je nach Methode und Kauterisiertem Gewebe deutlich unterscheiden kann.\\
        Hier sollte kritisch beurteilt und erprobt werden ob und in wie fern sich der Rauch unterschiedlicher Kauteriserungsmethoden unterscheidet. Weitere Erkenntnisse über den Rauch und seine Beschaffenheit können wertvolle Informationen bereitstellen welche die Implementierung der Erkennung und der Entfernung Verbessern oder vereinfachen können.\\
        
    \paragraph{Datenqualität}
        Eine wirkliche Beurteilung der erzeugten Daten fehlt. Um Sicherzustellen das der Datensatz auch den beschrieben Kriterien entspricht sollte er noch einem Mediziner vorgelegt werden, welche die ''Realität Aspekte'' bestätigen oder widerlegen sollte bzw. definiert welche teile der Daten adäquat sind und wo ggf. dinge sind die beachtet werden müssen.
        
    \subsection{Falsche Realität}
        opake flecken \\
        splitscreen - artzt muss denken
    \subsection{Skalierung}

\section{Verbesserungen}
    \subsection{Generelle Gedanken}
    Die Anforderungen für Verbesserungen sind in diesem Kontext bereits nicht trivial zu definieren. Denn die Aussage ein Video sei ''besser'' als ein anderes ist eine sehr persönliche Aussauge, die je nach Betrachter subjektiv bewertet wird. \\
    Eine verlässlichere Aussage über die Realität der erzeugten Daten sollte von einem Arzt vorgenommen werden. Andernfalls können nur sehr vage Aussagen getroffen werden. \\
    Der Versuch über Histogramme die Beschaffenheit der Daten zu erklären ist weitgehend unzureichend. \\
    Histogramme können Auskunft über die Farbzusammenstellung geben, jedoch gibt es Konstellationen und Muster die eine Aussage unmöglich machen.\
    So können zwei unterschiedliche Bilder über das gleiche Histogramm verfügen, ein einfaches Umsortieren der Pixel reicht aus um ein Bild unkenntlich zu machen, während dessen das andere Histogramm unverändert bleibt. \\
    \newline
    Gleiches gilt für bearbeitete Bilder, auch hier gilt es einen Experten zu befragen. \\
    \newline

    \paragraph{Rechtliches}
    Ein weiterer Punkt für die Veränderung intraoperativer Bilder ist die Haftung bei Schäden am Patienten.\\
    Jede Veränderung der originalen Eingabedatei zur Verbesserung ist ein Abändern der aufgenommenen Realität, so wäre es beispielsweise möglich das eine Fehlfunktion des Algorithmus dazu führt, dass durch die Bearbeitung des Videos kritische Verletzungen verschleiert werden oder etwa an gesunden Stellen der Chirurg zu einem unnötigen Schnitt verleitet wird. \\
    Dies führt zu einer rechtlich unklaren Situation, etwa wer die Haftung für etwaige Schäden übernehmen muss.\\
    Dies mag mitunter ein Punkt sein weshalb sich systeme wie ''Ultravision'' nicht durchgesetzt haben.

\subsection{Dampf und Spritzer}
    Bis jetzt nur wenig beachtet - Dampf - aber ein relativ großes Problem.
\subsection{Nutzen der Stereoskopie}
    Einfacher Strahlensatz

\subsection{Multispektrales Licht}
    Ein Gedanke der Früh gefasst, aber  auch früh verworfen wurde war die Möglichkeit die Lichtquellen des Endoskope mit LEDs auszustatten die in verschiedenen Spektren den rauch Belichten, dies könnte so schnell geschehen das es im Normalbetrieb für das Auge des Chirurgen nicht wahrnehmbar ist jedoch ein Bild erzeugt das den rauch aufgrund seiner spektralen Antwort sinnvoll identifizieren kann. \\
    Dies wäre eine Erweiterung zur bisherigen optischen Erkennung und würde sich mit wenig zusätzlicher Hardware realisieren lassen. Nach bisherigen abschätzungen wäre es Notwendig eine Kamera zu verwenden die die entsprechenden Spektren aufzeichnen kann und eine passende, synchronisierte Lichtquelle bereit zu stellen.\\
    Dieser Ansatz musste jedoch aufgrund von Finanzierungsproblemen, sowie einem unrealistischen ausmaß für eine Thesis fallengelassen werden.
    Dennoch ist dieser Ansatz vielversprechend und kann mit den vorgeschlagenen Methoden dieser Thesis vereinbart werden:\\
    Das identifizieren und des Rauches könnte so verbessert werden um dann mit den vorgeschlagenen Bildbearbeitungsalgorithmen kombiniert zu werden.

\subsection{Erkennung durch Neuronale Netze}
    Ein aktuell sehr nachgefragtes Thema ist die Mustererkennung mittels Neuronaler Netze \cite{CNN_Image}.\\

   Formen/Verhalten von Rauch / Dampf / Spritzern  einlernen\\

    Vergleichend lässt sich hier sagen dass die realen Daten \ref{fig:cumsum_real} heller sein müssen als die selbst generierten.\\
    Dies lässt sich durch die Tatsache untermauern das bei der Generation der Daten lediglich eine von zwei möglichen Lichtquellen an die Kamera angeschlossen wurden.
    In Ermangelung einer zweiten Lichtquelle war dies bereits bei der Generation der Daten abzusehen und entspricht daher den vorherigen Erwartungen.\\
    Die selbst generierten Daten sind im direkten Vergleich zu den realen Beispieldaten aus Günzburg deutlich dunkler, dies zeigen auch die kumulierten Histogramme der beiden Datensätze \ref{fig:Cmp_hists}

    Für die ersten Versuche dieser Reihe spielt die Helligkeit eine untergeordnete Rolle, denn Rauch und andere Verunreinigungen können auch in den dunkleren Bildern deutlich mit bloßem Auge erkannt werden.\\
    Da die unzureichende Lichtquelle bereits bei der Erstellung der Daten als Problem erkannt wurde führte dies dazu, dass die Aufnahmen der Organe mit geringen Abstand zum Objektiv aufgezeichnet wurden. \\

    \begin{figure}[h]
        \includegraphics[width=\textwidth]{pics/ownData/darkpic.jpg}
        \caption{Bild mit schlechtem Licht und deshalb zu nah aufgenommener Leber}
        \label{fig:closeup}
    \end{figure}

    Auch dies steht nicht im direkten Konflikt des Ziels Schmutz aufzuzeichnen, jedoch entsteht so ein deutlich sichtbarer Unterschied zu den realen Datensätzen.\\
    Für zukünftige Versuche ist jedoch zwingend an eine weitere oder bessere Lichtquelle zu denken, da dies zu den offensichtlichsten Unterschieden im Vergleich mit den realen Daten gehört. Darüber hinaus führt das fehlende Licht dazu dass das Endoskope sehr nah an die Organe und Werkzeuge gehalten werden muss. Der Platz, in dem sich der Rauch ausbreiten kann, ist begrenzt, die Einwirkungen des Dampfes sind wesentlich stärker und häufiger zu sehen. Ebenso ist die Wahrscheinlichkeit der Tropfenbildung an der Kamera durch Spritzer wesentlich größer.\\
    \newline

    Zu Erwähnen ist ausserdem die Bewegung der Werkzeuge, die in den selbst generierten Videos definitiv weniger professionell abläuft als in den realen Daten, was sich einfach durch mangelnde Erfahrung der ''Chirurgen'' des Versuches erklären lässt.\\
    Generell hat das keine negativen Auswirkungen auf die Qualität des Schmauches. Jedoch kommt es oft vor, dass das Kamerabild verrutscht, nicht direkt die kauterisierte Stelle zeigt, oder wackelt. \\
    Auch dies sind keine direkten Nachteile, allerdings würden es viele dieser Bilder in realen Operationen dem Chirurgen nicht erlauben seiner Arbeit ordnungsgemäß nachzugehen.

        \clearpage
\section{Ausblick}
    \subsection{Histogram Matching}
        Das Bild \ref{fig:smokeHisto} zeigt die Anwendung der Histogram-matching Algorithmen auf ein Bild und deren resultierende Histogrammkurven.\\
        \begin{figure}[H]
            \includegraphics[width=\textwidth]{pics/eval/smokeHist.jpg}
            \caption{Anpassungen der Bilder mittels Histogram-matching}
            \label{fig:smokeHisto}
        \end{figure}

    Histogram Matching ist eine Vorstufe des in des \gls{DCP} \cite{darkChannel},\cite{imageDehaze}.\\
    Da mit dieser arbeit die ersten Schritte in Richtung Bildbearbeitung bereits getan sind wäre der nächste schritt die nicht erreichte Erkennung des Rauches in Angriff zu nehmen. Mit der Erkennung des Rauches kann dann in Richtung des eigentlichen Ziels gearbeitet werden. \\
    Die Schritte der Rauchererkennung setzen mit großer Wahrscheinlichkeit voraus das von Bild Daten auf Videos gewechselt werden muss, weshalb es sicherlich sinnvoll ist die in Octave bzw. matlab programmierten Scripte effizient in Python oder C++ Code zu übersetzen, um hier sinnvoll mit der openCV-Library weiterarbeiten zu können.


