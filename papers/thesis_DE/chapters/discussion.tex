\paragraph{other aproaches}

\begin{itemize}
	\item jeder sagt immer irgendwas von wegen \textbf{\textit{DIATHERMIE}} auf jedenfall noch angucken - wie hängt der strom mit dem Rauch zusammen 

	\item LEDs mit verschiedenen Farben für andere spektrale absorbtion im Bild
    \item Stereoskop bilder -> kann helfen bei erkennung?
	\item ausserdem noch die Licht geschichte mit dem Motorrad helm beschlagen...
\end{itemize}

\paragraph{Medical Considerations}

\begin{itemize}
	\item Quality of Tests / Setups
		\begin{itemize}
			\item TestBox
			\item Capturing Methoden
			\item Tools
		\end{itemize}
	\item with respect to the Cauterization kinds - thermal / hf 
	\item Dr. Kunze - changes in smoke due to the carbonisation degree of the material
	\item Dr. Kunzes videos as extra eval data - to avoid overfitting
\end{itemize}

\paragraph{Technical Considerations}

\begin{itemize}
	\item everything that takes way to much time in \texttt{RL} situations 
	\item AI approaches
	\item DDS and single point of failure
	\item Scalability
\end{itemize}

\paragraph{Overall considerations}  

\begin{itemize}
        \item Verlgeich zu Pyhsikalischen methoden - warum sollte man es rechnen?
        \item Weitere überlegung bezüglich stereoskopischer bilder - beide benutzen, was "fliegt" ist rauszumachen
        \item Falsche Fakten! - das bild wird bearbeitet ! - nicht ECHT!
            \item bei Opaken flecken? - was machen 
            \item disskussion über maßnahmen
\end{itemize}
