
\section{Kritischer Blick auf die Arbeit}
    
\subsection{Ziele der Arbeit}
    \begin{table}[H]
        \centering
        \begin{tabular}{|l|l|}	%all left aligned
            \hline
            \textbf{Ziel} & \textbf{Status} \\
            \hline
            Generieren von Bildern die Störungen Enthalten & Erreicht \\
            Erkennen von Störungen & Nicht Erreicht \\
            Entfernen von Störungen & Teilweise Erreicht \\
            Evaluation im vergleich zu realen Bildern & Nicht Erreicht \\
            Evalutaion im Vergleich zu den eingangsbildern & Nicht Erreicht \\
            \hline
        \end{tabular}
        \caption{Ziele der Arbeit\label{tab:goals}}
    \end{table}

    Tabelle \ref{tab:goals} Listet die Ursprünglichen Ziele der Arbeit und deren Status. \\
    Aufgrund der Aufwändigen Generierung der Testdaten war es im Umfang der Arbeit nicht Möglich das eigentliche Ziel der Arbeit, eine live Verbesserung des Video Inputs zu erreichen.\\
    
\subsubsection{Erzeugung der Basisdaten}
    Der Zeitliche Aufwand war bei der Ursprünglichen Einschätzung des Zeitplans noch kein teil der Arbeit, da fälschlicher weise davon ausgegangen wurde dass Verrauchte Bilder öffentlich zugänglich sind bzw, leichter zu erlangen sind.\\
    Die Generation der Daten hat insgesamt fast 3 Monate der 6 Monatigen Arbeit eingenommen, die vorab nicht eingeplant waren. Jedoch sind die verunreinigten Eingabe Daten für die Entwicklung und Erprobung der Algorithmen unabdingbar.\\ 
    Insbesondere die Organisation der Medizinischen Geräte gestaltete sich schwer da diese zum einen Teil zur Hochschule Ulm bzw. dem DKFZ gehören (Endoskope) und zum anderen von der NewTec GmbH (Kauter \cite{sealer} und HF-Generator \cite{hfgen})im Rahmen von Projekten benötigt werden.\\
    Bürokratische Hürden machen eine Sinnvolle Kooperation unnötig schwer.\\
    \newline
    Ein weiteres unterschätztes Problem ist die Beschaffung der Organe. Organe in entsprechender Qualität, sprich frische und mit Besonderheiten, wie beispielsweise einer noch nicht entfernten Gallenblase sind schwer zu bekommen, da heutzutage nur noch weniger Metzger selbst schlachten und ihre Waren von nur wenigen Schlachthöfen bekommen.\\
    Um einen Reibungslosen Ablauf zu garantieren müssen die Versuche entsprechend gut koordiniert werden, denn am Versuchstag können kleine Planungsfehler verheerend auf das Ergebnis Einfluss nehmen.\\
    \newline

\subsection{Qualität der Video Daten}
    Die Qualität der Video Daten wurde bereits im laufe der Methodik mittels Algorithmen bestimmt, für einen Verlgeich wurden Histogramme zu 
    
\subsection{Was ist der Nutzen?}
    Der gesamt nutzen der Arbeit besteht in vollständigen Datensätzen die die weitergehende Entwicklung der Algorithmen ermöglichen. Diese Video Daten waren vorher nur in kleinen mengen verfügbar, im rahmen dieser Arbeit entstanden 75 Minuten Videomaterials. Besonders Augenmerk der Daten liegt in der Eignung für die Entwicklung von Algorithmen, beispielsweise zeigt sich dies daran das eing Großteil der Videos ohne Verschmutzungen startet die sich dann soweit ausbreiten das mit bloßem Auge nicht mehr damit gearbeitet werden kann. \\
    Diese Eigenschaft kann in er Erprobung der Algorithmen verwendet werden um festzustellen ab welchem grad es dem Algorithmus nicht mehr wirksam die Verschmutzungen erkennen bzw entfernen kann.

\section{Vergleiche}

\subsection{Vergleich zu Physikalischen Dispense Systemen}
    Die Entferung von Rauch im Pneumoperitoneum kann wie bereits in der Einführung angedeutet existiren bereits einige mehtoden die sich mit der thematik des Surgical smokes beschäftigen. Die meisten dieser systeme stützen sich jedoch auf hardware um zu verhindern dass das Endoskope überhaupt erst einmal beschlägt, bzw. sich der Rauch im Pneumoperitoneum ausbreitet.\\
    Hierfür gibt es sauger, die um die werkzeuge angebracht werden \ref{fig:succer} \cite{succer} oder sogenannte Electrical Dispense Systems, die die Rauchpartikel electrostatisch aufladen, sodass diese zu den Wänden des Pneumoperitoneums magnetisch angezogen werden.\\

    \begin{figure}[H]
        \centering
        \includegraphics[width=\textwidth]{pics/usable_unsure/succer.jpg}
        \caption{Absaugung von Surgical smoke,\\
            mit dem Ziel den Rauch an der bereits beim enstehen zu entfernen.
        }
        \label{fig:succer}
    \end{figure}

    Diese Systeme finden sich jedoch selten in der Realität, denn sie sind mit extra Aufwand bei der Operation verbunden.\\
    Sauger, etc bringen weitere Hardware in den Operationssaal und sowie unhandliche Kabel und Schläuche mit sich, was bereits als Problem diagnostiziert wurde, wärend der Hospitation im \gls{ADK} beim Gespräch mit Dr Lotspeich \cite{Lotspeich}.\\
    \newline
    \cmt{Electrical Dispense system}

\subsection{Software Lösungen}
\fuck{Alesi surgical Ultravision}

Sieht im video nicht schlecht aus ... gibt aber kaum quellen \\
unseriös weil Eingabedatei ist anderes Video als Ausgabedatei.

\section{Verbesserungen}
    \subsection{Generelle Gedanken}
    Die Anforderungen für Verbesserungen sind in diesem Kontext bereits nicht trivial zu definieren. Denn die aussage ein Video sei ''besser'' als ein anderes ist eine sehr persönliche aussauge die Je nach Betrachter subjektiv bewertet wird. \\
    Eine verlässlichere aussage über die Realität der Erzeugten Daten sollte von einem Arzt vorgenommen werden. Andernfalls können nur sehr vage aussagen getroffen werden. \\
    Der versuch über Histogramme die Beschaffenheit der Daten zu erklären ist weitgehend unzureichend. \\
    Histogramme können Auskunft über die Farbzusammenstellung geben, jedoch gibt es Konstellationen und Muster die eine aussage unmöglich machen.\
    So können 2 unterschiedliche Bilder über das gleiche Histogram verfügen, ein einfaches umsortieren der Pixel reicht aus um ein Bild unkenntlich zu machen, während dessen Histogram unverändert bleibt. \\
    \newline
    Gleiches gilt für bearbeitete Bilder, auch hier gilt es einen Experten zu befragen. \\
    \newline

    \paragraph{Rechtliches}
    Ein weiterer Punkt für die Veränderung intraoperativer Bilder ist die Haftung bei Schäden am Patienten.\\
    Jede Veränderung der Originalen Eingabe Datei zur Verbesserung ist ein Abändern der aufgenommenen Realität, so wäre es beispielsweise möglich das eine Fehlfunktion des Algorithmus dazu führt dass durch die Bearbeitung des Videos kritische Verletzungen verschleiert werden oder etwa an gesunden stellen der Chirurgen zu einem unnötigen schnitt verleitet wird. \\
    Dies führt zu einer Rechtlich unklaren situation wer die haftung für etwaige schäden übernehmen muss.\\
    Dies mag mit unter ein Punkt sein weshalb sich systeme wie ''Ultravision'' nicht durchgesetzt haben.

\subsection{Dampf und Spritzer}
    Bis jetzt nur wenig beachtet - Dampf aber relativ großes problem.   
\subsection{Nutzen der Stereoskopie}
    Einfacher strahlensatz

\subsection{Multispektrales Licht}
    Multispektrale Absorbtion

\subsection{Erkennung durch Neuronale Netze}
    Ein aktuell sehr nachgefragtes Thema ist die Mustererkennung mittels Neuronaler Netze \cite{CNN_Image}.\\

   Formen/Verhalten von Rauch / Dampf / spritzern  einlernen\\
    \cmt{Ausfürlicher gehts dann im Ausblick weiter}

\begin{itemize}
	\item jeder sagt immer irgendwas von wegen \textbf{\textit{DIATHERMIE}} auf jedenfall noch angucken - wie hängt der strom mit dem Rauch zusammen 
	\item LEDs mit verschiedenen Farben für andere spektrale absorbtion im Bild
    \item Stereoskop bilder -> kann helfen bei erkennung?
	\item ausserdem noch die Licht geschichte mit dem Motorrad helm beschlagen...
\end{itemize}

\paragraph{Medical Considerations}

\begin{itemize}
	\item Quality of Tests / Setups
		\begin{itemize}
			\item TestBox
			\item Capturing Methoden
			\item Tools
		\end{itemize}
	\item with respect to the Cauterization kinds - thermal / hf 
	\item Dr. Kunze - changes in smoke due to the carbonisation degree of the material
	\item Dr. Kunzes videos as extra eval data - to avoid overfitting
\end{itemize}

\paragraph{Technical Considerations}

\begin{itemize}
	\item everything that takes way to much time in \texttt{RL} situations 
	\item AI approaches
	\item DDS and single point of failure
	\item Scalability
\end{itemize}

\paragraph{Overall considerations}  

\begin{itemize}
        \item Verlgeich zu Pyhsikalischen methoden - warum sollte man es rechnen?
        \item Weitere überlegung bezüglich stereoskopischer bilder - beide benutzen, was "fliegt" ist rauszumachen
        \item Falsche Fakten! - das bild wird bearbeitet ! - nicht ECHT!
            \item bei Opaken flecken? - was machen 
            \item disskussion über maßnahmen
\end{itemize}
