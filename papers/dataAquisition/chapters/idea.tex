\subsection{Motivation}
	The motivation to create this kind of data is given by the fact that there are only few free and open videos of laparoscopic surgeries, that show smoke or other adverse effects on the camera, since they exactly counter what a surgeon usually wants to archive: \textit{ a clear vision on the patients internals}. This document therefore describes the creation of \textit{hazy, foggy, blurry and stained} endoscopic images and videos. 

\subsection{Primer}
	The objective is to create data in a custom setup, outside of the body of any creature. This series consists of a capturing setup and multiple iterations of data grabbing, with different documented changes to the environment as well as to the cauterized objects.\newline

The archived data will contain a set of \textbf{videos} and \textbf{images} that show \textit{noise}. This noise is defined by:\newline
	\paragraph{surgical smoke \newline} 
		Surgical smoke occurs when surgeons cauterize organs, veins or other body internal objects.
			Although it contents are mostly steam, there are multiple other particles dispensed within this kind of Fog. Mostly the leftovers of the cauterization given by the explosion of the cell. This releases all the cells contents, including tentative bacteria into into the pneumoperitoneum.  \cite{contentOfSmoke} \newline
	Depending on the time, type and tissue of the cauterization the smoke can vary in its aspect, also the carbonisation level of the tissue influences the smoke and its aspect. 


	\paragraph{haze \newline} 
		Haze is a effect that also appears alongside to cauterization, the heat of the surgical cauters leads to the vaporization of body liquids. This leads to fog, steam and haze that can block the direct vision to the targeted areas and also can build a thin film on the endoscopes objective, as well as \textit{staining} which shows up as the formation of drops on the objectives.

	\paragraph{staining \newline} 
		Multiple body internal fluids get directly in touch with the endoscopes objective. These fluids gather to build drops, which blur or cover certain parts of the cameras vision. \newline
\newline

\subsection{Objective}
All of the above mentioned effects are required to occur withing the generated test data, as it would be the case in a real surgery. 
The Quality of the images should be quite high, since typically used endoscopes in modern surgery often have high frame rates (above 60FPS) and resolutions of more than 4K2K (4000x2000 Pixels).\cmt{src?}
\newline
The produced data focuses on a setup that is as close as possible to a real surgery. \newline
The quality of the images is therefore just a secondary goal. More important is the creation of videos, as well as images that contain multiple combinations of the given pollutions. 
