\subsection{fundamental papers}

The Paper \cite{MIS_host} introduces a learning method for expectant surgeons to practice without living patients, be it human or animal. Since these surgeries are expensive, time consuming and moreover they need to be authorized by governmental or veterinarian institutions in the European Union. \newline
\newline

The data generation of surgical smoke and artefacts requires the use of organic material that can be cauterized. Porcine visceral organs are very similar to human visceral organs which the paper \cite{swineModel} postulates and are therefore a appropriate candidate for the test setups. \newline
\newline

Surgical smoke is different to ''normal'' smoke because of its contents and their composition as given by the paper \cite{contentOfSmoke}. Therefore the generation of the test data needs to be performed with a special focus on the realism of the smoke and haze. Also not only the cauterized organs and their humility are important to the aspect of the smoke. According to experts also the method as well as the duration of cauterization is an important influence on the smokes quality.  \newline
\newline

Laparoscopic trainers are often used in eduction and training of future surgeons. Therefor a source for the setup of these kind of trainers and how they should look like is taken from the papers \cite{MIS_host} and \cite{MIS_trainer}. \newline
\newline

\subsection{Requirements to the resulting data}
	The resulting data needs to fulfill several requirements, mostly concerning the occurrence and mixture of noise the as well as the quality of the images.\newline
	Investigated noises are:
	\begin{itemize}
		\item Surgical Smoke 
		\item Haze
		\item Staining 
		\begin{itemize}
			\item Blood
			\item Water
			\item Lipids
		\end{itemize}
	\end{itemize}
	With these problems in mind the most critical of them need to be selected and further investigated. 

\subsection{ Selection }
	The following section deals with pre-testing thoughts on the Organ / Animal selection 
	\paragraph{ Host Animals }
		\begin{enumerate}
			\item domestic Pig ( Sus scrofa domesticus ) \textit{referred as Porcine}
			\item cattle ( Bos primigenius taurus ) \textit{referred as Bovine}
			\item house Mouse ( Mus musculus)	
		\end{enumerate}

		The enumeration given above lists possible candidate animals for the organs. Due to practical reasons only the 3 of them are getting into a closer selection. \newline
\newline
		Mice have a quite high similarity to the human body, but problems occur in the availability and the handling of these organs. Therefore Mice's organs drop out, since they are expected to have a bad handling due to their small size. \newline
		\newline
		Porcine Organs fit better for this purpose, but especially their internals are not very commonly consumed. For closer testings it is expected that porcine organs are necessary, while for first setup tests or technical evaluations other organs might be sufficient.\newline
		\newline
		Bovine Organs are the most commonly available visceral organs, since they are often used in culinary way. Tripes or livers especially. Although their similarity to human internals is not that high, their advantages in availability and usability makes them a good choice for first experiments.\newline
		\newline	
		As given in \cite{swineModel}, the common domestic pig has already often been used as the major species for animal testing of both pharmaceutical and surgical experiments. Due to the fact that pigs have a lot of similarities to the human body, e.g. the resemblance of their cardiovascular systems or the digestive systems.\newline
		Concluding to this, the data should contain a set of video footage that is based on the cauterization of porcine internals.

	\paragraph{ Organs }
		The most laparoscopic surgeries are performed in the area of the abdomen, so the most reasonable things to use as target objects for the cauterization would include:
		\begin{itemize}
			\item Gallbladder ( Vesica fellea )
			\item Liver (Hepar)
			\item Stomach ( Astomagus )
			\item Gastrointestinal tract ( Tractus digestorius )
		\end{itemize}

		The Selection of Organs depends on similar choices as the selection of animals. The decision is based on their availability and usability, countering to real surgeries like liver resections or Cholecystectomies.\newline
		Another point is the internal moisture that is given by mucous membranes, their cauterisation leads to fog and steam. This issue needs to be considered since moisture is expected to create foggier images that therefore are closer to real surgeries within wet environments.\newline
		This moisture occurs in forms like Blood, lipid and other body internal fluids. The heating of these lipids generates different kinds of smokes, hazes and fogs, depending on the lipid.\newline
		Based on this information, the insertion of these lipids needs to be considered within the setups.
